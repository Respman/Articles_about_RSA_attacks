\documentclass[12pt,a4paper]{scrartcl}
\usepackage[utf8]{inputenc}
\usepackage[english,russian]{babel}
\usepackage{indentfirst}
\usepackage{misccorr}
\usepackage{graphicx}
\usepackage{amsmath}
\usepackage{hyperref}
\usepackage{float}

\begin{document}

\section{Мультипликативное свойство RSA}

Если внимательно присмотреться к RSA, то можно заметить, что система обладает  мультипликативным свойством, то есть

$$RSA(m_1)*RSA(m_2)=RSA(m_1*m_2),$$

потому что 

$$m_1^e*m_2^e = (m_1*m_2)^e\ (mod\ n)$$

Иногда данное свойство ещё называют гомоморфностью RSA относительно умножения.

Этот факт можно применить следующим образом: если у вас есть шифрование открытых текстовых чисел 2, 3 и 5, то мы сможем создать каждый зашифрованный текст, который может быть составлен как произведение этих чисел, например 4, 8, 16, 32 ..., или 6, 9, 12, 15, 18, 24... 

\end{document}
