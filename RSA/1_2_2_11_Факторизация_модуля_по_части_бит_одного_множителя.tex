\documentclass[12pt,a4paper]{scrartcl}
\usepackage[utf8]{inputenc}
\usepackage[english,russian]{babel}
\usepackage{indentfirst}
\usepackage{misccorr}
\usepackage{graphicx}
\usepackage{amsmath}
\usepackage{hyperref}
\usepackage{float}
\usepackage{minted} %для вставки кода в документ

\begin{document}

\section{Факторизация модуля $N$ по частичной информации о $N/4$ старших битах $p$ и известной нижней оценке $p$}

Если нам известен порядок одного из множителей модуля $N$ $\beta$ такой, что $p > N^\beta$, то мы сможем восстановить множитель $p$, имея чуть больше чем $log_2(N)/4$ верхних бит числа $p$.

Для этого нам нужно применить \href{https://yatb.kksctf.ru/}{метод Копперсмита} к многочлену следующего вида:

$$f(x) = p_0 + x\ (mod\ p),$$

где $p$ - один из множителей модуля $N$, а $p_0$ - известная нам верхняя часть бит.\\

\textbf{Sage-реализация:}

\inputminted[tabsize=4,obeytabs,fontsize=\footnotesize]{python3}{./RSA_scripts/upper_bit_p.sage}

\textbf{Источник информации об атаке факторизацией модуля $N$ по частичной информации о $N/4$ старших битах $p$ и известной нижней оценке $p$:}\\

\href{https://github.com/mimoo/RSA-and-LLL-attacks/blob/master/survey_final.pdf}{https://github.com/mimoo/RSA-and-LLL-attacks/blob/master/survey\_final.pdf}

Don Coppersmith. 1996. Finding a Small Root of a Bivariate Integer Equation;Factoring with High Bits Known. InProceedings of EUROCRYPT’96. Springer-Verlag, 178–189

\href{https://github.com/mimoo/RSA-and-LLL-attacks/blob/master/survey_final.pdf}{https://github.com/mimoo/RSA-and-LLL-attacks/blob/master/survey\_final.pdf}

\href{https://crypto.stanford.edu/~dabo/papers/RSA-survey.pdf}{https://crypto.stanford.edu/~dabo/papers/RSA-survey.pdf}

Alexander May \textit{Using LLL-Reduction for Solving RSA and Factorization Problems}

\section{Факторизация модуля $N$ по частичной информации о $N/4$ нижних битах $p$ и известной нижней оценке $p$}

Данная атака аналогична изложенной ранее атаке \href{https://yatb.kksctf.ru/}{факторизации модуля $N$ по частичной информации о $N/4$ старших битах $p$ и известной нижней оценке $p$}, только теперь мы изменим уравнение на 

$$f(x,y)=2^kx+p_0,$$

потому что нам теперь известны не верхние, а нижние биты множителя $p$.\\

\textbf{Sage-реализация:}

\inputminted[tabsize=4,obeytabs,fontsize=\footnotesize]{python3}{./RSA_scripts/lower_bit_p.sage}

\textbf{Источник информации об атаке факторизацией модуля $N$ по частичной информации о $N/4$ младших битах $p$ и известной нижней оценке $p$:}

\href{https://github.com/mimoo/RSA-and-LLL-attacks/blob/master/survey_final.pdf}{https://github.com/mimoo/RSA-and-LLL-attacks/blob/master/survey\_final.pdf}

Don Coppersmith. 1996. Finding a Small Root of a Bivariate Integer Equation;Factoring with High Bits Known. InProceedings of EUROCRYPT’96. Springer-Verlag, 178–189

\href{https://github.com/mimoo/RSA-and-LLL-attacks/blob/master/survey_final.pdf}{https://github.com/mimoo/RSA-and-LLL-attacks/blob/master/survey\_final.pdf}

\href{https://crypto.stanford.edu/~dabo/papers/RSA-survey.pdf}{https://crypto.stanford.edu/~dabo/papers/RSA-survey.pdf}

Alexander May \textit{Using LLL-Reduction for Solving RSA and Factorization Problems}

\end{document}
