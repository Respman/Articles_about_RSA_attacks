\documentclass[12pt,a4paper]{scrartcl}
\usepackage[utf8]{inputenc}
\usepackage[english,russian]{babel}
\usepackage{indentfirst}
\usepackage{misccorr}
\usepackage{graphicx}
\usepackage{amsmath}
\usepackage{hyperref}
\usepackage{float}
\usepackage{minted} %для вставки кода в документ

\begin{document}

\section{Квадратичное решето}

Метод квадратичного решета является модификацией метода разложения на множители Диксона, который в свою очередь является обобщением метода Ферма. Данный метод является универсальным для атаки "грубой силой".

Сложность расчёта для квадратичного решета ( $n$ - факторизуемое число):

\begin{center}
	$O(exp((1+o(1)){\sqrt {\log n\log \log n}}))$
\end{center}

\textbf{Подход}\\

Пусть $x\ mod\ y$ обозначает остаток от деления $x$ на $y$. В методе факторизации Ферма в отдельности подбираем число $a$, чтобы $a^2\ mod\ n$ являлось квадратом (потому что мы вычитаем $n$, поэтому мы и рассматриваем сравнение по модулю $n$). Но такое число подобрать тяжело. В квадратичном решете мы вычисляем $a^2\ mod\ n$ для некоторых $a$, и затем находим такое подмножество, произведение элементов которого является квадратом. Это приведёт к сравнению квадратов.

Например, $41^2\ mod\ 1649 = 32,\ 42^2\ mod\ 1649 = 115,$ и $43^2 mod\ 1649 = 200$. Ни один из полученных результатов не является квадратом, но результат произведения $(32)(200) = 6400 = 80^2$. С другой стороны, рассмотрев предыдущее произведение по $mod\ 1649$, мы получим, что $(32)(200) = (412)(432) = ((41)(43))^2$. Так как $(41)*(43)\ mod\ 1649 = 114$, мы имеем сравнение квадратов: $114^2 \equiv 80^2 (mod 1649)$.

Но как мы найдём подмножество, произведение элементов которого является квадратом? Начнём с понятия вектор показателей степеней. Например, у нас есть число $504$. Его разложение на простые множители имеет следующий вид $504 = 2^33^25^07^1$. Мы могли бы представить это разложение в виде вектора показателей степеней $(3,2,0,1)$, который фиксирует степени простых чисел $2$, $3$, $5$ и $7$, участвующих в разложении. Число $490$ по аналогии имело бы вектор $(1,0,1,2)$. Умножение чисел - это то же самое, что и поэлементное сложение их векторов показателей степеней, то есть произведение $504*490$ имеет вектор $(4,2,1,3)$.

Теперь, обратите внимание, что число является квадратом, если каждый элемент в его векторе показателей степеней чётный. К примеру, при сложении векторов $(3,0,0,1)$ и $(1,2,0,1)$ получаем $(4,2,0,2)$. Это говорит нам о том, что произведение чисел $56*126$ является квадратом. На самом деле мы не заботимся о точных значениях, получаемых в векторе, до тех пор, пока каждый элемент в результирующем векторе чётный. По этой причине мы берём каждый элемент по $mod\ 2$ и выполняем сложение элементов по $mod\ 2$: $(1,0,0,1) + (1,0,0,1) = (0,0,0,0)$.

Таким образом, наша задача приняла следующий вид: задано множество векторов из элементов $\{0,1\}$, и нам нужно найти такое подмножество, которое дополняется до нулевого вектора, при использовании сложения по $mod\ 2$. Это задача линейной алгебры, то есть необходимо найти линейно зависимые вектора. Из теоремы линейной алгебры следует, что, до тех пор, пока количество векторов больше, чем количество элементов в каждом векторе, такая зависимость должна существовать. Мы можем эффективно находить линейно зависимые векторы, например, поместив исходные векторы, в качестве столбцов матрицы и затем использовать метод Гаусса, который легко приспособить для работы с целыми числами по $mod\ 2$. Как только мы найдём линейно зависимые векторы, мы просто перемножаем числа, соответствующие этим векторам, и получаем квадрат, который ищем.

Однако возведение в квадрат множества случайных чисел по $mod\ n$ приводит к большому числу различных простых множителей, длинным векторам и большой матрице. Чтобы избавиться от этой проблемы, мы специально ищем числа, такие, что $a^2\ mod\ n$ имеет только небольшие простые множители (такие числа называются гладкими числами). Их сложнее найти, но использование таких чисел позволяет избежать больших векторов и матриц.\\

\textbf{Алгоритм}

\begin{enumerate}
	\item Выбираются границы $P$ и $A$ порядка величины $e^{\sqrt {\log {n}\log {\log {n}}}}$ (далее обозначается как $L(n)$), такие что $P<A<P^{2}$.
	
	\item Для $t=[{\sqrt {n}}]+1,\ \left[{\sqrt {n}}\right]+2,...,\ \left[{\sqrt {n}}\right]+A$ по порядку в столбец выписываются целые числа $t^{2}-n$.
	
	\item Для каждого нечётного простого числа $p\leq P$ вычисляется символ Лежандра, то есть проверяется условие $\left({n \over p}\right)=1$.(Данное условие можно проверить с помощью \href{https://yatb.kksctf.ru/}{критерия Эйлера}). Если оно не выполняется, $p$ удаляется из факторной базы.
	
	\item Предполагая, что $p$ — такое нечётное простое число, что $n$ — квадратичный вычет по модулю $p$ (то есть данное число является квадратом другого числа в данном конечном поле), решается уравнение $t^{2}\equiv n{\pmod {p^{\beta }}}$ для $\beta = 1,2,...$. Значения $\beta$ берутся в порядке возрастания пока не окажется, что уравнение не имеет решений $t$, сравнимых по модулю $p^{\beta }$ с каким-либо из чисел в области $\left[{\sqrt {n}}\right]+1\leq t\leq \left[{\sqrt {n}}\right]+A$.
	
	Пусть $\beta$  — наибольшее из таких чисел, для которых в указанной области найдётся число $t$ со свойством $t^{2}\equiv n{\pmod {p^{\beta }}}$.
	
	Пусть $t_{1}$ и $t_{2}$ решения $t^{2}\equiv n{\pmod {p^{\beta }}}$ и $\displaystyle t_{2}\equiv -t_{1}{\pmod {p^{\beta }}}$.
	
	\item При том же значении $p$ просматривается список значений $t^{2}-n$, полученный в п.2. В столбце, соответствующем $p$, ставится 1 против всех значений $t^{2}-n$, для которых $t$ отличается от $t_{1}$ на некоторое кратное $p$. После этого 1 заменяется на 2 для всех таких значений $t^{2}-n$, что $t$ отличается от $t_{1}$ на кратное $p^{2}$ и т. д. до $\beta$. Затем то же самое проделывается с $t_{2}$ вместо $t_{1}$. Наибольшее возможное число в столбце — $\beta$.
	
	\item При добавлении очередной единицы к числу в столбце в п.5, соответствующее число $t^{2}-n$ делится на $p$ и полученный результат сохраняется.
	
	\item В столбце под $p=2$ при $n\neq 1{\pmod {8}}$ просто ставится 1 против $t^{2}-n$ с нечетным $t$ и соответствующее $t^{2}-n$ делится на 2. При $n\equiv 1{\pmod {8}}$ решается уравнение $t^{2}=n{\pmod {2^{\beta }}}$ и решение продолжается в том же ключе, как при нечетном $p$.
	
	\item Когда все указанные действия будут проведены для всех простых чисел, не превосходящих $P$, следует отбросить все числа $t^{2}-n$, кроме обратившихся в 1 после деления на все степени $p$, не превосходящих $P$. В итоге получится таблица, в которой в $b_{i}$ столбце будут содержаться все такие значения $t$ из интервала $\left[{\left[{\sqrt {n}}\right]+1,\left[{\sqrt {n}}\right]+A}\right]$, что $t^{2}-n$ есть $B$-число, а остальные столбцы будут соответствовать тем значениям $p\leq P$, для которых $n$ — квадратичный вычет.
	
	\item Далее используется обобщенный метод факторизации Ферма (метод факторных баз).
	
	Всё, что мы делали до этого момента - искали среди всех чисел $B$-гладкие числа. Теперь с их помощью мы сможем найти такое произведение подмножества этих чисел, что оно будет являться полным квадратом. Поиск осуществляется следующим образом:
	
	Мы раскладываем каждое число из имеющихся на простые множители в степенях, после чего мы каждому простому числу до $B$ предоставляем столбик в матрице. Строками в данной матрице будут являться вектора степеней, в которых простые числа входят в разложение выбранного для строки числа. Для сформированной матрицы $A$ нам нужно найти нетривиальное решение матричного уравнения $xA = 0$, при этом элементы матрицы $A$ будут заданы по модулю 2. Конкретное нетривиальное решение $x$ и будет указывать на числа, перемножив которые мы получим полный квадрат.
	
\end{enumerate}

\textbf{Пример}\\

Этот пример демонстрирует стандартное квадратичное решето без логарифмических оптимизаций. Допустим, нам нужно факторизовать число $N = 15347$, следовательно, наименьшее число, квадрат которого больше $N$, равно 124. Значит $y(x) = (x + 124)^2 - 15347$.\\

\textbf{Сбор данных}\\

Так как $N$ мало, то нам необходимо только 4 простых числа. Первые 4 простых числа $p$, для которых у 15347 есть квадратный корень по модулю $p$, равны 2, 17, 23, и 29 (Другими словами, 15347 является квадратичным вычетом для этих простых чисел). Эти числа будут базисом для квадратичного решета.

Сейчас мы строим наше решето $V_{X}$ из $Y(X)=(X+\lceil {\sqrt {N}}\rceil )^{2}-N=(X+124)^{2}-15347$ и начнём с просеивания для каждого простого числа в базисе, выбирая для просеивания первые числа $Y(X),\ 0 \leq X < 100$:

$$ {\begin{aligned}V&={\begin{bmatrix}Y(0)&Y(1)&Y(2)&Y(3)&Y(4)&Y(5)&\cdots &Y(99)\end{bmatrix}}\\&={\begin{bmatrix}29&278&529&782&1037&1294&\cdots &34382\end{bmatrix}}\end{aligned}}$$

Следующим шагом является выполнение просеивания. Для каждого $p$ в нашей факторной базе $\lbrace 2,17,23,29\rbrace$ решаем уравнение

$$Y(X)\equiv (X+\lceil {\sqrt {N}}\rceil )^{2}-N\equiv 0{\pmod {p}}$$

чтобы найти элементы в массиве $V$, которые делятся на $p$.

Для $p=2$ решаем $(X+124)^{2}-15347\equiv 0{\pmod {2}}$ чтобы получить решение $X\equiv {\sqrt {15347}}-124\equiv 1{\pmod {2}}$.

Таким образом, начиная с $X=1$ с шагом 2, каждая запись будет делиться на 2. Деление каждой из записей на 2 даёт

$$V={\begin{bmatrix}29&139&529&391&1037&647&\cdots &17191\end{bmatrix}}$$

Аналогично для оставшихся простых чисел $p$ в $\lbrace 17,23,29\rbrace$ равенство $X\equiv {\sqrt {15347}}-124{\pmod {p}}$ решено. Обратите внимание, что для каждого $p > 2$ будет 2 результирующих линейных уравнения, из-за наличия двух квадратных корней по модулю.

$${\begin{aligned}X&\equiv {\sqrt {15347}}-124&\equiv 8-124&\equiv 3{\pmod {17}}\\&&\equiv 9-124&\equiv 4{\pmod {17}}\\X&\equiv {\sqrt {15347}}-124&\equiv 11-124&\equiv 2{\pmod {23}}\\&&\equiv 12-124&\equiv 3{\pmod {23}}\\X&\equiv {\sqrt {15347}}-124&\equiv 8-124&\equiv 0{\pmod {29}}\\&&\equiv 21-124&\equiv 13{\pmod {29}}\\\end{aligned}}$$

Каждое равенство $X\equiv a{\pmod {p}}$ приводит к тому, что $V_{x}$ делится на $p$ при $x=a,\ a+p,\ a+2p,$ и т.д.. Делением $V$ на $p$ в $a,\ a+p,\ a+2p,\ a+3p,$ и т.д., для каждого простого числа в базисе находим гладкие числа.

$$V={\begin{bmatrix}1&139&23&1&61&647&\cdots &17191\end{bmatrix}}$$

Элемент из $V$, который равен 1, соответствует гладкому числу. Так как $V_{0}$, $V_{3}$, и $V_{71}$ равняются 1, то:


\begin{table}[H]
	\centering
	\begin{tabular}{|l|l|l|}
		\hline
		\multicolumn{1}{|c|}{\textit{\textbf{$X+124$}}} & \multicolumn{1}{c|}{\textit{\textbf{$Y$}}} & \multicolumn{1}{c|}{\textbf{factors}} \\ \hline
		124                                             & 29                                         & $2^0 * 17^0 * 23^0 * 29^1$            \\ \hline
		127                                             & 782                                        & $2^1 * 17^1 * 23^1 * 29^0$            \\ \hline
		195                                             & 22678                                      & $2^1 * 17^1 * 23^1 * 29^1$            \\ \hline
	\end{tabular}
\end{table}

Далее мы применим метод факторных баз к нашим гладким числам.\\

\textbf{Обработка матрицы}\\

Рассмотрим равенства:

$${\begin{aligned}29&=2^{0}\cdot 17^{0}\cdot 23^{0}\cdot 29^{1}\\782&=2^{1}\cdot 17^{1}\cdot 23^{1}\cdot 29^{0}\\22678&=2^{1}\cdot 17^{1}\cdot 23^{1}\cdot 29^{1}\\\end{aligned}}$$

и выпишем показатели простых чисел в виде матрицы и решим систему ${\pmod {2}}$:

$$S\cdot {\begin{bmatrix}0&0&0&1\\1&1&1&0\\1&1&1&1\end{bmatrix}}\equiv {\begin{bmatrix}0&0&0&0\end{bmatrix}}{\pmod {2}}$$

Её решение:

$$S={\begin{bmatrix}1&1&1\end{bmatrix}}$$

Таким образом, произведение всех 3-х уравнений есть квадрат $(mod\ N)$.

$$29\cdot 782\cdot 22678=22678^{2}$$

и

$$124^{2}\cdot 127^{2}\cdot 195^{2}=3070860^{2}$$

алгоритм находит

$$22678^{2}\equiv 3070860^{2}{\pmod {15347}}$$

Теперь вычисляем НОД$(3070860 - 22678, 15347) = 103$, значит нетривиальным делителем числа 15347 является число 103, а другим является 149.\\ 

\textbf{Источники информации про квадратичное решето:}

\href{https://ru.wikipedia.org/wiki/%D0%9C%D0%B5%D1%82%D0%BE%D0%B4_%D0%BA%D0%B2%D0%B0%D0%B4%D1%80%D0%B0%D1%82%D0%B8%D1%87%D0%BD%D0%BE%D0%B3%D0%BE_%D1%80%D0%B5%D1%88%D0%B5%D1%82%D0%B0}{https://ru.wikipedia.org/wiki/Метод\_квадратичного\_решета}
	
\href{http://dha.spb.ru/PDF/cryptoFACTOR.pdf}{http://dha.spb.ru/PDF/cryptoFACTOR.pdf}\\
	
Мы решили описать для вас именно этот алгоритм, хотя он и является вторым по быстроте, потому что самый быстрый алгоритм  слишком сложен в объяснении и дает заметный выигрыш только на очень больших числах длинной более 110 десятичных знаков - это агоритм "Общий метод решета числового поля". Данный метод является продолжением метода квадратичного решета с тем отличием, что просеивание производится не по простым числам, а по многочленам специального вида.\\
	
Далее мы приведем пример реализации методов квадратичного решета на языке sage - реализовывать метод Ферма и метод Полларда не имеет смысла, потому что метод квадратичного решета уже встроен в sage и будет работать с той же скоростью, что и эти два метода.\\
	
\textbf{Sage-реализация:}
	
\inputminted[tabsize=4,obeytabs,fontsize=\footnotesize]{python3}{./RSA_scripts/recheto.sage}
	
\textbf{Источник информации про реализации методов факторизации на языке sage:}\\
	
\href{https://doc.sagemath.org/html/en/thematic_tutorials/explicit_methods_in_number_theory/integer_factorization.html}{https://doc.sagemath.org/.../integer\_factorization.html}\\

\end{document}
