\documentclass[12pt,a4paper]{scrartcl}
\usepackage[utf8]{inputenc}
\usepackage[english,russian]{babel}
\usepackage{indentfirst}
\usepackage{misccorr}
\usepackage{graphicx}
\usepackage{amsmath}
\usepackage{hyperref}
\usepackage{float}
\usepackage{minted} %для вставки кода в документ

\begin{document}

\section{Маленькая экспонента d: атака Бонеха-Дюрфи (Boneh-Durfee attack)}

Данная атака позволяет получить закрытую экспоненту $d$, если $d< N^{0.292}$ - достаточно мало.

Рассмотрим цепочку преобразований:

$$\parbox{8cm}{$ed \equiv 1\ (mod\ \phi(N)) \\
	\rightarrow ed=k*\phi(N)+1 \\
	\rightarrow k*\phi(N)+1 \equiv 0\ (mod\ e) \\
	\rightarrow k*(p-1)*(q-1)+1 \equiv 0\ (mod\ e) \\
	\rightarrow k*(N+1-p-q)+1 \equiv 0\ (mod\ e) \\
	\rightarrow k*(\frac{N+1}{2}+\frac{-p-q}{2})+1 \equiv 0\ (mod\ e)$}$$

Теперь если мы обозначим $\frac{N+1}{2}=A,\ \frac{-p-q}{2}=y,\ 2k=x$, тогда мы получим многочлен от двух переменных: 

$$f(x,y)=1+x(A+y)$$

Корни этого многочлена помогут нам вычислить закрытую экспоненту $d$.

Мы могли бы использовать \href{https://yatb.kksctf.ru/}{метод Копперсмита} и могли бы получить оценку $d < N^{0.284}$, но для данного конкретного уравнения существует достаточно красивая оптимизация, которая позволяет поднять оценку эффективности алгоритма до $d < N^{0.292}$, поэтому мы сразу рассмотрим оптимизированный алгоритм.

Для успешной атаки должны соблюдаться условия: $d < N^{\delta},\ |x|< 2N^{\delta},\  |y| < N^{0.5},\ \delta < 1-\dfrac{\sqrt{2}}{2} \approx 0.292,\ t=\tau m=(1-2\delta)m,\ m$ подбирается по условию, которое юудет описано ниже. Для начала можно поставить $m=4$.

Для начала проведём замену переменной $u=xy+1$, чтобы линеаризовать многочлен:

$$\overline{f}(x,u)=u+Ax\ (mod\ e)$$

Построим полиномы  для $k=0,...,m:$

$$\overline{g}_{i,k}(u,x)=x^i*\overline{f}^k(u,x)*e^{m-k}\text{ для }k=0,...,m\text{ и }i=0,...,m-k$$
$$h_{j,k}(u,x,y)=y^j*\overline{f}^k(u,x)*e^{m-k}\text{ для }j=0,...,t\text{ и }k=\lfloor\dfrac{m}{t}\rfloor,...,m$$

Которые называются $x$-сдвигом ($\overline{g}_{i,k}$) и $y$-сдвигом ($\overline{h}_{j,k}$).

Собираем из них матрицу:

$$B = \begin{pmatrix}
	f_1(xX,yY)\\
	...\\
	f_n(xX,yY)
\end{pmatrix}=$$

\begin{center}
	\begin{table}[H]
		
		\begin{tabular}{cc}
			
			& 1\ \qquad$x$\ \qquad$u$\ \qquad$x^2$\ \qquad$ux$\ \qquad$u^2$\ \qquad$u^2y$\\
			
			\begin{tabular}[c]{@{}c@{}}$e^2$\\\\ $xe^2$\\\\ $\overline{f}e$\\\\ $x^2e^2$\\\\ $x\overline{f}e$\\\\ $\overline{f}^2$\\\\ $y\overline{f}^2$\end{tabular} & \begin{tabular}[c]{@{}c@{}}$\begin{pmatrix}\\ e^2&&&&&&\\\\ &e^2X&&&&&\\\\ &eAX&eU&&&&\\\\ &&&e^2X^2&&&\\\\ &&&e^2AX^2&eUX&&\\\\ &&&A^2X^2&2AUX&U^2&\\\\ &-A^2X&-2AU&&A^2UX&2AU^2&U^2Y\\\\ \end{pmatrix}$\end{tabular}
			
		\end{tabular}
		
	\end{table}
\end{center}

Проверяем условие, которое накладывается на определитель матрицы:

$$det(B)<e^{mn}$$

Если вдруг получилось так, что это условие не выполняется, увеличиваем значение $m$ и пересобираем матрицу $B$.

Далее на матрицу $B$ применяется \href{https://yatb.kksctf.ru/}{LLL-алгоритм}.\\

$$B = \begin{pmatrix}
	b_1=g_1(xX,yY)\\
	b_2=g_2(xX,yY)\\
	...\\
	b_n
\end{pmatrix}$$

После работы LLL-алгоритма в получившейся матрице находим два линейно-независимых вектора и представляем их как коэффициенты многочленов:

$$g_1(x_0,y_0)\equiv 0\ (mod\ e^m)\text{ где } |g_1(xX,yY)|<\dfrac{e^m}{\sqrt{n}}$$
$$g_2(x_0,y_0)\equiv 0\ (mod\ e^m)\text{ где } |g_2(xX,yY)|<\dfrac{e^m}{\sqrt{n}}$$

Так как выполняются условия теоремы Копперсмита, то мы из уравнений над конечными полями $(mod\ e^m)$ можем перейти к полю целых чисел и решать уравнения над ним.

Получаем
$$g_1(x_0,y_0) = 0$$
$$g_2(x_0,y_0) = 0$$

То есть в итоге мы получили систему из двух уравнений от двух переменных - так как уравнениями мы взяли линейно независимые вектора из сокращённого LLL-алгоритмом базиса, то значит у системы есть единственное решение.

Как итог получаем многочлен $r(x)=resultant_x(g_1,g_2)$, корни которого можно найти любым численным методом, например комбинацией \href{https://yatb.kksctf.ru/}{алгоритма Берлекемпа} и \href{https://yatb.kksctf.ru/}{линейного подъёма Гензеля}.

После чего мы можем вычислить $k = x/2$, решить сравнение $k*\phi(N)+1\equiv 0\ (mod\ e)$ и в конечном итоге получить $d$, используя \href{https://yatb.kksctf.ru/}{расширенный алгоритм Евклида}.

\textbf{Sage-реализация:}\\

\inputminted[tabsize=4,obeytabs,fontsize=\footnotesize]{python3}{./RSA_scripts/boneh_Durfee.sage}

\textbf{Источник:}

\href{https://github.com/mimoo/RSA-and-LLL-attacks/blob/master/survey_final.pdf}{https://github.com/mimoo/RSA-and-LLL-attacks/blob/master/survey\_final.pdf}

\href{https://www.cryptologie.net/article/241/implementation-of-boneh-and-durfee-attack-on-rsas-low-private-exponents/}{https://www.cryptologie.net/article/241/implementation-of-boneh-and-durfee-attack-on-rsas-low-private-exponents/}

\href{https://crypto.stanford.edu/~dabo/papers/RSA-survey.pdf}{https://crypto.stanford.edu/~dabo/papers/RSA-survey.pdf}

\end{document}
