\documentclass[12pt,a4paper]{scrartcl}
\usepackage[utf8]{inputenc}
\usepackage[english,russian]{babel}
\usepackage{indentfirst}
\usepackage{misccorr}
\usepackage{graphicx}
\usepackage{amsmath}
\usepackage{hyperref}
\usepackage{float}

\begin{document}

\section{Атака Копперсмита на разницу двух открытых текстов (Coppersmith’s short pad attack)}

Метод атаки Копперсмита на разницу двух открытых текстов основан на следующих действиях нарушителя: 

Абоненты при шифровании сообщения могут дополнять его некоторым фиксированным паддингом из $m$ случайных бит на одном из концов сообщения:

$$M' = 2^mM+r$$

Если при таком паддинге одно и то же сообщение будет послано два раза (например, атакующий может перехватить сообщение и не передавать его получателю - абонент в таком случае вышлет это сообщение ещё раз, но с другим паддингом, что и даст возможность выполнить данную атаку.)

$$M_1 = 2^mM+r_1$$
$$M_2 = 2^mM+r_2$$

Для применения атаки мы составляем два многочлена:

$$g_1(x,y)=x^e-C_1$$
$$g_2(x,y)=(x+y)^e-C_2,$$

где $y=r2-r1$ - корень результата $h(y)=res_x(g_1,g_2) \in \mathbb{Z}_N[y]$. Степень h не более $e^2$. Более того, $|y|< 2^m<N^{1/e^2}$, то есть $y$ является 'маленьким' корнем многочлена $h(y)$ по модулю N, а значит его можно вычислить \href{https://yatb.kksctf.ru/}{методом Копперсмита}. Если известно решение $y$, то из $M$ можно восстановить $M_1$ с помощью \href{https://yatb.kksctf.ru/}{атаки Франклина-Рейтера}, представив $M_2=M_1+y$ и составив зависимость $f(x)=x+y$.\\

\textbf{Источники:}

\href{https://crypto.stanford.edu/~dabo/papers/RSA-survey.pdf}{https://crypto.stanford.edu/~dabo/papers/RSA-survey.pdf}

\href{http://honors.cs.umd.edu/reports/lowexprsa.pdf}{http://honors.cs.umd.edu/reports/lowexprsa.pdf} 

\href{https://github.com/yud121212/Coppersmith-s-Short-Pad-Attack-Franklin-Reiter-Related-Message-Attack/blob/master/coppersmiths_short_pad_attack.sage}{https://github.com/.../coppersmiths\_short\_pad\_attack.sage}

\end{document}
