\documentclass[12pt,a4paper]{scrartcl}
\usepackage[utf8]{inputenc}
\usepackage[english,russian]{babel}
\usepackage{indentfirst}
\usepackage{misccorr}
\usepackage{graphicx}
\usepackage{amsmath}
\usepackage{hyperref}
\usepackage{float}
\usepackage{minted} %для вставки кода в документ

\begin{document}

\section{BB attack - атака Бляйхенбахера}

Данная атака ориентированна на паддинг PKCS\#1 1.5.

В простом параллельном канале чётности (атака pairing oracle) мы видели, как использовать мультипликативное свойство RSA для взлома зашифрованного текста, как только чётность открытого текста становилась нам известна. Утечка одного бита позволяет восстановить всё сообщение.  Каждая утечка чётности позволяет разбить интервал возможных открытых текстов на два подынтервала, размер которых отличается не более чем на единицу, поэтому алгоритм, используемый в оракле чётности является оптимальным (строит сбалансированное двоичное дерево поиска).

Атака Блайхенбахера аналогична pairing oracle, но основана на несколько ином побочном канале, который мы будем называть padding oracle. Атака срабатывает всякий раз, когда возвращается ошибка заполнения PKCS\#1 v1.5, сообщающая, что расшифрованное сообщение не соответствует схеме заполнения. Атака, по сути, использует случаи, в которых допускается заполнение, поскольку они показывают, что первые два байта являются 0x00 и 0x02.

Итак, обозначим за $k$ длину модуля в битах, тогда сообщение вместе с рандомной частью будет иметь $B=2^{k-16}$ возможных вариантов, при этом если учесть, что первые 2 байта сообщения это '0x00' и '0x02', то мы можем сказать, что любое сообщение+паддинг для модуля данной длины будет лежать в интервале $[2B,3B-1]$. Тут можно провести аналогию с интервалом для атаки ораклом чётности.

Следующим нашим шагом будет поиск такого числа $s_1$, для которого сообщение 

$$s_1^e*m_0^e=(s_1*m_0)^e(mod\ N) = m_1^e(mod\ N)$$

Расшифруется с корректным паддингом (то есть система примет ваше сообщение $m_1$ как сообщение с паддингом PKCS\#1 v1.5). Поиск данного числа начинается с какого-то фиксированного уровня (в нашем случае с $\lceil \dfrac{N}{3B}\rceil$) и просто инкрементируется, пока мы не получим нужное $s_1$, которое удовлетворяет условию выше.

Найдя подходящее $s_1$, мы можем уменьшить интервал для $m_0$, составив следующее равенство:

$$m_0*s_1=m_1+r*N$$
$$m_0=\dfrac{m_1+r*N}{s_1},$$

где $r$ - неизвестно.

В итоге мы можем составить границы для $m_0$:


$$2*B \leq m_1 \leq 3*B-1, \rightarrow$$
$$\dfrac{2*B+rN}{s_1} \leq m_0 \leq \dfrac{3*B-1+rN}{s_1}, \rightarrow [r=\dfrac{m_0s_1-m_1}{N}] \rightarrow$$
$$\dfrac{2Bs_1-3B+1}{N} \leq r \leq \dfrac{(3B-1)s_1-2B}{N}$$

(мы не можем оперировать в конечном виде формулы значениями $m_0$ и $m_1$, так как мы их не знаем).

Теперь мы выбираем все возможные значения $r$, удовлетворяющие условию выше, и для каждого $r$ мы рассчитываем интервал (потому что мы не знаем, сколько раз было совершено приведение по модулю, поэтому рассмотрим все возможные случаи):

$$\dfrac{2B+rN}{s_1} \leq m_0 \leq \dfrac{3B-1+rN}{s_1}$$

Каждый получившийся интервал пересекаем с интервалом $[2B,3B-1]$.

В результате мы получили несколько интервалов, в одном из которых должно содержаться исходное сообщение $m_0$. Для того, чтобы нам выяснить, где именно оно находится, мы находим следующее $s_i$ таким же способом, как и $s_1$, только начиная перебор со значения $s_{i-1}$.

После нахождения $s_i$ мы сможем уточнить информацию для каждого из интервалов, полученных ранее. Для значения $s_i$ верно утверждение (по аналогии с $s_1$):

$$\dfrac{2B+rN}{s_i} \leq m_0 \leq \dfrac{3B-1+rN}{s_i}$$

Теперь рассчитаем информацию о новых возможных $r$, используя интервалы, которые мы уже имеем (то есть теперь мы будем уточнять не интервал $[2B,3B-1]$, а все интервалы $[a, b]$, полученные на предыдущем шаге):

Выведем формулу границ для $r$ на интервале $[a, b]$ из формулы:

$$\dfrac{2Bs_1 - 3B + 1}{n} \leq r \leq \dfrac{(3B-1)s_1 - 2B}{n},$$

заменяя $2Bs_i$ на $as_i$ и $(3B-1)s_i$ на $bs_i$.

$$\dfrac{as_i - 3B + 1}{n} \leq r \leq \dfrac{bs_i - 2B}{n}$$

Вычисленные новые интервалы для $m_0$ пересекаются с интервалом $[a, b]$.

Если в результате уточнения всех имеющихся интервалов мы получили больше одного промежутка, то мы находим новое $s_i$ и повторяем процедуру снова. 

Для случая, когда у нас остался один промежуток $[a, b]$ для $m_0$, мы можем выполнить важную оптимизацию, которая делает атаку сходящейся очень быстро. Мы опишем её ниже.


\textbf{Бинарный поиск с одним оставшимся интервалом}

Из формулы для новых интервалов для $m_0$ на шаге $i$ 

$$\dfrac{2B+rN}{s_i} \leq m_0 \leq \dfrac{3B-1+rN}{s_i}$$

мы непосредственно выводим

$$\dfrac{2B+rN}{m_0} \leq s_i \leq \dfrac{3B-1+rN}{m_0}$$

Поскольку $m_0$ находится в $[a, b]$ (у нас есть только один возможный интервал), мы получаем ограничение пространства поиска для $s_i$

$$\dfrac{2B+rN}{b} \leq s_i \leq \dfrac{3B-1+rN}{a}$$

если мы выберем $r \geq \dfrac{2(bs_i-2B)}{N}$, то получим

$$\dfrac{2B+2(bs_{i-1} - 2B)}{b} \approx 2s_{i-1} \leq s_i$$

Таким образом, новая $s_i$ по меньшей мере вдвое превосходит предыдущую ступень. Обратите внимание, что все новые интервалы для $m_0$ на шаге $i$

$$\dfrac{2B+rN}{s_i} \leq m_0 \leq \dfrac{3B-1+rN}{s_i}$$

имеют размер $B/s_i$. Как следствие, удвоение $s_{i-1}$ даёт интервал, который в два раза меньше интервала предыдущего шага. Это напоминает двоичный поиск, который мы реализовали с помощью побочного канала чётности, и заставляет атаку сходиться в несколько шагов, которые являются линейными по количеству битов.\\

Получая 'бинарным поиском' новые значения интервала $[a, b]$, мы его уменьшаем и в конечном счёте он сходится к интервалу нулевой длины, границами которого и будет являться дешифрованное сообщение.


Для ускорения поиска $s_i$ мы можем домножать $m_0^e$  на малые дроби (например, на $\dfrac{19}{17}$) - то есть мы умножаем $m_0^e$ на маленькую дробь а надежде, что старшие биты не изменятся. Но в нашей программной реализации мы не использовали данную оптимизацию (чтобы читателю было проще разобраться в написанном коде).\\

\textbf{Sage-реализация:}

\inputminted[tabsize=4,obeytabs,fontsize=\footnotesize]{python3}{./RSA_scripts/bleichenbacher.sage}

\textbf{Источник:}

\href{http://secgroup.dais.unive.it/wp-content/uploads/2012/11/Practical-Padding-Oracle-Attacks-on-RSA.html}{http://secgroup.dais.unive.it/...Oracle-Attacks-on-RSA.html}

\href{https://www.youtube.com/watch?v=ibfb7\_-\_CGg\&list=PLLguubeCGWoaGFEDzduGmBhEgZ62p-Jqv\&index=6}{https://www.youtube.com/watch?...\&index=6}\\


\end{document}
