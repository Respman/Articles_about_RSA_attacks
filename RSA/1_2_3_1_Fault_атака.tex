\documentclass[12pt,a4paper]{scrartcl}
\usepackage[utf8]{inputenc}
\usepackage[english,russian]{babel}
\usepackage{indentfirst}
\usepackage{misccorr}
\usepackage{graphicx}
\usepackage{amsmath}
\usepackage{hyperref}
\usepackage{float}
\usepackage{minted} %для вставки кода в документ

\begin{document}

\section{Fault-атака}

При программной реализации алгоритма RSA иногда предпочитают проводить вычисления не с модулем $n$, а прямо с числами $p$ и  $q$  - это позволяет ускорить вычисления и сделать реализацию эффективнее. А помогает в этом \href{https://yatb.kksctf.ru/}{Китайская теорема об остатках}. Применяя данную теорему на практике, можем получить следующую конструкцию:

$$\left.\parbox{3.5cm}{$s_{1} = m^{d_p}\ (mod\ p) \\ s_{2} = m^{d_q}\ (mod\ q)$}  \right\} \rightarrow s\ (mod\ pq)$$

Иногда данная оптимизация может сыграть злую шутку: допустим, мы хотим использовать такую ускоренную реализацию для цифровой подписи: в этом режиме сообщение 'подписывается' на закрытом ключе, а 'проверяется' целостность сообщения на открытом ключе.

Но если в вычислениях (например, $s_{2}$) происходит ошибка, то мы получаем следующий результат при подписании сообщения:

$$\left.\parbox{3.5cm}{$s_{1} = m^{d_p}\ (mod\ p) \\ \widetilde{s_{2}} = m^{d_q}\ (mod\ q)$}  \right\} \rightarrow \widetilde{s}\ (mod\ pq)$$

Теперь у нас есть неправильно подписанное сообщение, которое обладает следующими свойствами при его "проверке":

$$\begin{cases} \widetilde{s^e} = m\ (mod\ p)\\ \widetilde{s^e} \neq m\ (mod\ q) \end{cases} \rightarrow \begin{cases} \widetilde{s^e} - m = 0\ (mod\ p)\\ \widetilde{s^e} -m \neq 0\ (mod\ q) \end{cases} \rightarrow \begin{cases} p | \widetilde{s^e} - m \\ q \nmid \widetilde{s^e} -m \end{cases} $$

То есть $\widetilde{s^e} - m = pk$, где $k$ - какое-то число. Поэтому если мы посчитаем НОД$(\widetilde{s^e} - m, N)$, то мы получим НОД$(\widetilde{s^e} - m, N)=p$. Таким образом мы разложили модуль $N$ - теперь нам не составит труда найти $d$, зная $e$.\\

\textbf{Sage-реализация:}

\inputminted[tabsize=4,obeytabs,fontsize=\footnotesize]{python3}{./RSA_scripts/fault_attack.sage}

\textbf{Источник:}

\href{https://www.cryptologie.net/article/371/fault-attacks-on-rsas-signatures/}{https://www.cryptologie.net/article/371/fault-attacks-on-rsas-signatures/}

\end{document}
