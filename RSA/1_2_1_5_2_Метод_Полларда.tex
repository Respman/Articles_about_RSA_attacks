\documentclass[12pt,a4paper]{scrartcl}
\usepackage[utf8]{inputenc}
\usepackage[english,russian]{babel}
\usepackage{indentfirst}
\usepackage{misccorr}
\usepackage{graphicx}
\usepackage{amsmath}
\usepackage{hyperref}
\usepackage{float}

\begin{document}

\section{(P-1) - метод Полларда}

(P-1) - метод Полларда хорошо действует если $p-1$, $q-1$ или $p+1$, $q+1$ не имеют большого простого делителя (являются гладкими).

Данный алгоритм привел к появлению понятия сильного простого числа, используемого в криптографии - простого числа, для которого $p-1$ имеет достаточно большие делители. В современных криптосистемах стараются использовать именно сильные простые числа, так как это повышает стойкость используемых алгоритмов и систем в целом.

Число называется $B$-гладкостепенным, если все его простые делители, в степенях, в которых они входят в разложение этого числа $p^{\nu }$, удовлетворяют $p^{\nu }\leq B$. 


\textbf{Первая стадия:}

\begin{enumerate}
	\item Пусть $N$ $B$-гладкостепенное, и требуется найти делитель числа $N$. В первую очередь вычисляется число $M(B)=\prod _{i}p_{i}^{k_{i}}$ где произведение ведётся по всем простым $p_{i}$ в максимальных степенях $k_{i}:p_{i}^{k_{i}}<B$
	\item Тогда искомый делитель $q=\text{НОД}(b-1,N)$, где $b=a^{M(B)}\mod N$, $\text{НОД}(a, N) =1$.
\end{enumerate}

Возможно два случая, в которых приведенный выше алгоритм не даст результата:

\begin{enumerate}
	\item В случае, когда $\text{НОД}(b-1,N)=N$ точно можно сказать, что у $N$ есть делитель, являющийся $B$-гладкостепенным и проблему должен решить иной выбор $a$.
	\item В более частом случае, когда $\text{НОД}(b-1,N)=1$ стоит перейти ко второй стадии алгоритма, которая значительно повышает вероятность результата, хотя и не гарантирует его.\\
\end{enumerate}


\textbf{Пример применения 1-ой стадии:}\\

Пусть $N = 10001$ выберем $B = 10$, тогда $M(B)=2^{3}\cdot 3^{2}\cdot 5\cdot 7=2520$, возьмём $a=2$ и вычислим теперь $b=a^{M(B)}\mod N=2^{2520}\mod 10001=3578$, и наконец $(b-1,N)=\text{НОД}(3578-1,10001)=73$.\\

\textbf{Замечания}\\

При больших $B$ число $M(B)$ может оказаться весьма большим, сравнимым по значению с $B!$, в таких случаях может оказаться целесообразно разбить $M(B)$ на множители приблизительно одинаковой величины $M(B)=\prod _{i}M_{i}$ и вычислять последовательность

\begin{center}
	$a_{1}=a^{M_{1}}\mod N;$\\
	$...$\\
	$a_{k+1}=a_{k}^{M_{k+1}}\mod N$.\\
\end{center}

После чего перемножив полученные результаты: $a=a_{1}*...*a_{k+1}$\\

\textbf{Вторая стадия:}

Прежде всего необходимо зафиксировать границы $\displaystyle B_{1}=B,\ B_{2}\gg B$, обычно $B_{2}\leq B^{2}$.

Вторая стадия алгоритма находит делители $N$, такие что $p-1=q\cdot A$, где $A$ — $B$-гладкостепенное, а $q$ - простое, такое что $B_{1}<q<B_{2}$.

\begin{enumerate}
	\item Для дальнейшего нам потребуется вектор из простых чисел $q_{i}$ от $B_1$ до $B_{2}$, из которого легко получить вектор разностей между этими простыми числами $D=(D_{1},D_{2},...),D_{i}=q_{i+1}-q_{i}$, причём $D_{i}$ — относительно небольшие числа, и $D_{i}\in \Delta$, где $\Delta$ — конечно множество. Для ускорения работы алгоритма полезно предварительно вычислить все $b^{\delta _{i}},\forall \delta _{i}\in \Delta$ и пользоваться уже готовыми значениями.
	\item Теперь необходимо последовательно вычислять $c_{0}=b_{1}\mod N,\ c_{i}=c_{i-1}^{\delta _{i}}\mod N$, где $b_{1}=a^{M(B_{1})}\mod N$, вычисленное в первой стадии, на каждом шаге считая $G=\text{НОД}(c_{i}-1,N)$. Как только $G\neq 1$, можно прекращать вычисления.\\
\end{enumerate}


\textbf{Условия успешной отработки алгоритма}\\

\begin{itemize}
	\item Пусть $p$ наименьший делитель $N$, $q^{t}=max(q_{i}^{t_{i}})$, а максимум берётся по всем степеням $q_{i}^{t_{i}}$, делящим $p-1$.
	\item Если $q^{t}<B_{1}$, то делитель будет найден на первой стадии алгоритма.
	\item В противном случае для успеха алгоритма необходимо, чтобы $q^{t}<B_{2}$, а все остальные делители $p-1$ вида $q^{r}$ были меньше $B_1$.\\
\end{itemize}


\textbf{Оценка эффективности}\\

Сложность первой стадии оценивается как $O(B_{1}\cdot \ln B_{1}\cdot (\ln N)^{2})$.

Сложность второй стадии оценивается как $O({\frac {B_{2}}{\ln B_{2}}})$.\\

\textbf{Источник:}\\

\href{https://ru.wikipedia.org/wiki/P%E2%88%921-%D0%BC%D0%B5%D1%82%D0%BE%D0%B4_%D0%9F%D0%BE%D0%BB%D0%BB%D0%B0%D1%80%D0%B4%D0%B0}{https://ru.wikipedia.org/wiki/P-1-метод\_Полларда}
	
\href{https://algolist.manual.ru/maths/teornum/factor/p-1.php}{https://algolist.manual.ru/maths/teornum/factor/p-1.php}
	
\href{http://window.edu.ru/resource/845/23845/files/book.pdf}{http://window.edu.ru/resource/845/23845/files/book.pdf}

\end{document}
