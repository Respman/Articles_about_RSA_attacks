\documentclass[12pt,a4paper]{scrartcl}
\usepackage[utf8]{inputenc}
\usepackage[english,russian]{babel}
\usepackage{indentfirst}
\usepackage{misccorr}
\usepackage{graphicx}
\usepackage{amsmath}
\usepackage{hyperref}
\usepackage{float}

\begin{document}

\section{"Слепое" подписывание сообщения}

Данная атака основана на мультипликативном свойстве RSA:

Иногда требуется получить подпись на конкретное сообщение у системы, которая этого делать не хочет. В случае, если в роли подписи выступает алгоритм RSA,  то возможно получить подпись на сообщение даже без согласия на это самой системы. Делается это следующим образом:

Изначально у нас есть сообщение $M$, которое требуется подписать, но система делать это отказывается. Для этого мы можем выбрать случайное $r \in \mathbb{Z}^*_N$, собираем конструкцию следующего вида: $M^\prime =r^eM\ mod N$ и просим систему подписать сообщение $M^\prime$:

$$S^\prime = (M^\prime)^{d}\ mod\ N$$

Но, если мы распишем $M^\prime$, то получим следующую конструкцию:

$$S^\prime=(M^\prime)^{d}\ mod\ N = (r^eM)^{d}\ mod\ N = r^{ed}M^{d}\ mod\ N = rM^{d}\ mod\ N$$

То есть чтобы получить подписанное сообщение $M$, нам просто нужно разделить $S^\prime$ на $r$.\\

\textbf{Источник:}

\href{https://crypto.stanford.edu/~dabo/papers/RSA-survey.pdf}{https://crypto.stanford.edu/~dabo/papers/RSA-survey.pdf}

\end{document}
