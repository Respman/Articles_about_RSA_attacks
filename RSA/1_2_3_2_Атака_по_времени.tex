\documentclass[12pt,a4paper]{scrartcl}
\usepackage[utf8]{inputenc}
\usepackage[english,russian]{babel}
\usepackage{indentfirst}
\usepackage{misccorr}
\usepackage{graphicx}
\usepackage{amsmath}
\usepackage{hyperref}
\usepackage{float}
\usepackage{minted} %для вставки кода в документ

\begin{document}

\section{Атака по времени}

Данная атака является атакой на реализацию криптографического алгоритма RSA на устройствах с ограниченной вычислительной мощностью (например, на смарт-картах). 

Для шифрования в таких устройствах используются реализации \href{https://yatb.kksctf.ru/}{алгоритма быстрого возведения в степень}.

Допустим, мы знаем, что всего в двоичной записи экспоненты у нас $\omega$ бит. Если мы знаем первые $b-1$ бит экспоненты, то мы сможем посчитать $b-1$ её первых шагов. Так как мы знаем характеристики смарт-карты, то мы сможем просчитать заранее время $t$, которое потребуется карточке для возведения сообщения $M_i$ в степень $d_0d_1...d_{b-1}$. Переменной $T$ мы обозначим время возведения числа $M$ в степень, равную полной экспоненте $d=d_0d_1...d_{\omega}$. Мы не знаем, чему будет равняться $d_{b}$, но мы имеем всего два возможных случая ($d_{b}=1$ или $d_{b}=0$). Если мы за $D(\xi)$ обозначим \href{https://yatb.kksctf.ru/}{дисперсию случайной величины} $\xi$, то мы можем посчитать $D(\{T\}-\{t\}) = (w-(b-1))\cdot D(t)$ для $i$ штук открытых сообщений $M$ с экспонентой, равной $d_0d_1...d_{b-1}$, а после этого посчитать величину $D(\{T\}-\{t\}) = (w-b)\cdot D(t)$ для такого же количества сообщений, но с экспонентой $d_0d_1...d_{b}$ при $d_{b}=1$. Если дисперсия для экспоненты $d_0d_1...d_{b}$ больше дисперсии с экспонентой $d_0d_1...d_{b-1}$, то значит предположение неверно и $d_{b}=0$; если же дисперсия уменьшится, то гипотеза о том, что $d_{b}=1$, верна.\\

Первый бит экспоненты всегда единица, иначе бы мы не смогли найти для закрытой экспоненты открытую. Поэтому мы сразу принимаем $d_{0}=1$.

Данный алгоритм является вероятностным, поэтому нам нужно рассчитать количество запросов, при которой мы сможем доверять нашей оценке. Кочер (Kocher) в своей статье вычисляет нужное значение $i$ по следующей формуле:  $\Phi\left(\sqrt{\dfrac{i(b-c)}{2(\omega - b)}}\right)$. Здесь $\Phi(x)$ - интеграл по функции плотности нормального распределения от $-\infty$ до $x=\sqrt{\dfrac{i(b-c)}{2(\omega - b)}}$, $b$ - количество имеющихся бит, $c$ - количество верных известных бит экспоненты $d$ (в нашем случае $c=b-1$), а $\omega$ - количество бит в закрытой экспоненте.

Но при проведении эксперимента проще поступить следующим образом - если нам известно общее количество бит в закрытой экспоненте, то мы можем подобрать опытным путём для любой другой экспоненты того же размера (при том же модуле) количество запросов $i$, при котором у нас правильно восстанавливается загаданная нами экспонента. После чего просто используем найденное значение $i$ для вычисления неизвестной экспоненты.

Измерение обычного времени работы алгоритма для нас слишком неточное, поэтому прийдётся рассчитывать время работы вручную: для этого мы можем эмулировать действия умножения и деления (взятие остатка можно выразить через деление, просто не используя целую часть от частного, а используя только остаток) как машинное \href{https://yatb.kksctf.ru/}{умножение} и \href{https://yatb.kksctf.ru/}{деление} на арифметическом логическом устройстве (АЛУ). Все элементарные операции (сложение, сдвиг на одну позицию, сравнение, присваивание) мы будем считать выполняющимися за один такт. В моем скрипте рассчитывается время, которое будет затрачено на выполнение умножения и деления в двоичном виде. Чтобы не усложнять код , мы немного блефуем и делаем операцию деления средствами языка - это сделано для сокращения кода скрипта. 

В одной из статей было сказано, что нам достаточно знать $N/4$ нижних бит экспоненты, чтобы потом восстановить оставшиеся другим методом; но метод, который использовался в статье, имеет линейную сложность относительно $N$ (там присутствует квадратное уравнение с "плохими" характеристиками, которые не позволяют применять к нему некоторые методы быстрого нахождения корней квадратного уравнения над конечным кольцом, поэтому остаётся только брут-форс по всем элементам кольца, что при больших $N$ является вычислительно сложной задачей).\\

\textbf{Sage-реализация:}

\inputminted[tabsize=4,obeytabs,fontsize=\footnotesize]{python3}{./RSA_scripts/timing_attack.sage}

\textbf{Источники:}

\href{https://crypto.stanford.edu/~dabo/papers/RSA-survey.pdf}{https://crypto.stanford.edu/~dabo/papers/RSA-survey.pdf}

P.Kocher. Timing attacks on implementations of Dffie-Hellman, RSA ,DSS , and other systems. InCRYPTO'96, volume1109 of Lecture Notes in Computer Science, pages 104-113. Springer-Verlag, 1996.

\end{document}
