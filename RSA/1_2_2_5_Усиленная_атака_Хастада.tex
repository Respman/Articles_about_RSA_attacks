\documentclass[12pt,a4paper]{scrartcl}
\usepackage[utf8]{inputenc}
\usepackage[english,russian]{babel}
\usepackage{indentfirst}
\usepackage{misccorr}
\usepackage{graphicx}
\usepackage{amsmath}
\usepackage{hyperref}
\usepackage{float}
\usepackage{minted} %для вставки кода в документ

\begin{document}

\section{Усиленная атака Хастада}

Чтобы защититься от обычной атаки, отправитель может добавлять к сообщению паддинг - дополнительные биты, сгенерированные специальным образом:
если само сообщение имело длину $m$-бит, то линейный паддинг будет выглядеть следующим образом:

$$f_i(M) = i 2^m + M$$

Хастад доказал, что добавление линейного паддинга не защищает систему от атаки его методом.

Допустим, каждый из $k$ участников имеет свой фиксированный многочлен $i$, который является его паддингом.

В таком случае одинаковое сообщение $M$  дополняется паддингом $f_i(M)$ и шифруется на ключе $e_i$ (ключи могут различаться):

$$C_i = f_i(M)^{e_i}\ mod\ N_i$$

для всех $i=1...k$. Все модули $N_i$ также попарно взаимно простые (аналогично вышесказанному).

Теперь нам нужно сформировать систему многочленов $g_i$ таких, что верно $g_i(M)\equiv 0\ (mod\ N_i)$ (они имеют $M$  как корень):


$$\begin{cases} g_1 = f_1(x)^{e_1} - C_1\ mod\ N_1\\ g_2 = f_2(x)^{e_2} - C_2\ mod\ N_2\\...\\ g_k = f_k(x)^{e_k} - C_k\ mod\ N_k\\ \end{cases}$$

Теперь нам нужно выполнить такой же переход, как и в обычной атаке Хастада: то есть мы по Китайской теореме об остатках найдём многочлен $g(x)$ по общему большому модулю $N_1N_2...N_k$ и сможем применить к нему \href{https://yatb.kksctf.ru/}{метод Копперсмита}.

Мы с помощью \href{https://yatb.kksctf.ru/}{Китайской теоремы об остатках} ищем коэффициенты $T_i$ такие, что $T_i \equiv 1\ (mod\ N_i)$ и $T_i \equiv 0\ (mod\ N_j)$ для всех $i\neq j$. То есть мы для каждого $T_i$ оставляем систему из $k$ сравнений (по двум условиям из предыдущего предложения) и таким образом находим этот коэффициент. Общий многочлен будет иметь вид:

$$g(x)= \sum T_i*g_i(x)$$

Максимальная степень полинома $g(x)$ у нас равна $max_i(deg\ (f_i(x)^{e_i}))=max_i(e_i*deg\ f_i(x))$ (потому что мы просто сложили все многочлены $g_i(x)$ вместе, помножив их на коэффициенты $T_i$), а значит нам нужно, чтобы корень этого многочлена, возведённый в степень $max_i(e_i*deg\ f_i(x))$, был много меньше, чем произведение $N_1N_2...N_k$ (по аналогии с обычной атакой Хастада). Поэтому нам необходимо, чтобы $k\geq max_i(e_i*deg\ f_i(x))$. Хочется обратить внимание на то, что при линейном паддинге $deg\ f_i(x)=1$, а значит и $k\geq max_i(e_i)$.

Хочется сделать маленькую оговорку - для метода Копперсмита нам требуется, чтобы многочлен $g(x)$ был унитарным, или хотя бы его старший коэффициент был обратим по модулю $N$. Но мы не можем это гарантировать со 100\%-ной вероятностью, так как модуль не является примарным числом (то есть не является простым числом в какой-то ненулевой степени). При одинаковых экспонентах у всех абонентов вероятность того, что коэффициент будет обратим, близка к 100\% (если сложить все коэффициенты $T_i$ вместе, то вероятнее всего получится обратимый элемент: так как каждый $T_i$ не делится на $N_i$ по построению, то мы получили, что каждое слагаемое не делится на какой-то из множителей $N=N_1N_2...N_k$. А вероятность того, что все они сложатся и дадут число, кратное модулю, не выше, чем вероятность обратимости суммы такого же количества случайных чисел, меньших $N$). Но если абоненты имеют различные экспоненты $e_i$, то нужно прибегнуть к одной хитрости - мы каждый из многочленов $g_i(x)$ помножим на унитарный многочлен $h(x)$, который будет иметь степень $deg(h(x)) = e_{max}-e_i$. Тогда все $g_i(x)$ будут иметь степень $e_{max}$, и мы сможем гарантировать с высокой долей вероятности, что многочлен $g(x)$ в итоге будет унитарный.\\

\textbf{Sage-реализация:}

\inputminted[tabsize=4,obeytabs,fontsize=\footnotesize]{python3}{./RSA_scripts/hastad_extend.sage}

\textbf{Источник:}

\href{https://crypto.stanford.edu/~dabo/papers/RSA-survey.pdf}{https://crypto.stanford.edu/~dabo/papers/RSA-survey.pdf}

\href{https://pdfs.semanticscholar.org/51a8/0f6eafb9d366bf729a44acff42313eaec041.pdf}{https://pdfs.semanticscholar.org/51a8/0f6eafb9d366bf729a44acff42313eaec041.pdf}

\href{https://en.wikipedia.org/wiki/Coppersmith%27s_attack}{https://en.wikipedia.org/wiki/Coppersmith\%27s\_attack}

\end{document}
