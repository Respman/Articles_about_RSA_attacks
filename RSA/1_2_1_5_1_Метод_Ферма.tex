\documentclass[12pt,a4paper]{scrartcl}
\usepackage[utf8]{inputenc}
\usepackage[english,russian]{babel}
\usepackage{indentfirst}
\usepackage{misccorr}
\usepackage{graphicx}
\usepackage{amsmath}
\usepackage{hyperref}
\usepackage{float}

\begin{document}

\section{Метод Ферма}

Данный метод используется, если числа p и q находятся достаточно близко друг к другу на числовой прямой.

Для разложения на множители нечётного числа $n$ ищется пара чисел $(x, y)$ таких, что $x^{2}-y^{2}=n$ (по формуле разности квадратов $(x-y)\cdot (x+y)=n$). При этом числа $(x+y)$ и $(x-y)$ являются множителями $n$, возможно, тривиальными (то есть одно из них равно 1, а другое — $n$.)

В нетривиальном случае, равенство $x^{2}-y^{2}=n$ равносильно $x^{2}-n=y^{2}$, то есть тому, что $x^{2}-n$ является квадратом.

Поиск квадрата такого вида начинается с $x=\left\lceil {\sqrt {n}}\right\rceil$  — наименьшего числа, при котором разность $x^{2}-n$ неотрицательна.

Для каждого значения $k\in \mathbb {N}$, начиная с $k=1$, вычисляют $(\left\lceil {\sqrt {n}}\right\rceil +k)^{2}-n$ и проверяют, не является ли это число точным квадратом. Если не является, то $k$ увеличивают на единицу и переходят на следующую итерацию.

Если $(\left\lceil {\sqrt {n}}\right\rceil +k)^{2}-n$ является точным квадратом, то есть $x^{2}-n=(\left\lceil {\sqrt {n}}\right\rceil +k)^{2}-n=y^{2}$, то получено разложение:

$n=x^{2}-y^{2}=(x+y)(x-y)=a\cdot b$, в котором $x=(\left\lceil {\sqrt {n}}\right\rceil +k)$

На практике значение выражения на $(k+1)$-ом шаге вычисляется с учётом значения на $k$-ом шаге:

$\left(s+1\right)^{2}-n=s^{2}+2s+1-n$, где $s=\left\lceil {\sqrt {n}}\right\rceil +k$.\\

\textbf{Числовой пример}\\

Пусть $n=89755$. Тогда ${\sqrt {n}}\approx 299,591$ или $s=\left\lceil {\sqrt {n}}\right\rceil =300$.

\begin{table}[h]
	\centering
	\begin{tabular}{|l|l|l|}
		\hline
		\multicolumn{1}{|c|}{\textbf{$k$}} & \multicolumn{1}{c|}{\textbf{$y$}} & \multicolumn{1}{c|}{\textbf{$\sqrt{y}$}} \\ \hline
		77                                 & 52374                             & 228,854                                  \\ \hline
		78                                 & 53129                             & 230,497                                  \\ \hline
		79                                 & 53886                             & 232,134                                  \\ \hline
		80                                 & 54645                             & 233,763                                  \\ \hline
		81                                 & 55406                             & 235,385                                  \\ \hline
		82                                 & 56169                             & 237                                      \\ \hline
	\end{tabular}
\end{table}

\begin{center}
	$\sqrt {y}=237$
	$a=s+k+{\sqrt {y}}=300+82+237=619$
	$b=s+k-{\sqrt {y}}=300+82-237=145$
\end{center}

Данное разложение является не конечным, так как, очевидно, что число $145$ не является простым: $=29\cdot 5.$

В итоге, конечное разложение исходного числа $n$ на произведение простых множителей $89755=5\cdot 29\cdot 619$.\\

\textbf{Оценка производительности}\\

Метод факторизации Ферма будет работать не хуже метода перебора делителей, если $Iter\ (n)<n^{1/2}$, отсюда можно получить оценку для большего делителя $a<4n^{1/2}$.\\


\textbf{Источник:}

\href{https://ru.wikipedia.org/wiki/%D0%9C%D0%B5%D1%82%D0%BE%D0%B4_%D1%84%D0%B0%D0%BA%D1%82%D0%BE%D1%80%D0%B8%D0%B7%D0%B0%D1%86%D0%B8%D0%B8_%D0%A4%D0%B5%D1%80%D0%BC%D0%B0}{https://ru.wikipedia.org/wiki/Метод\_факторизации\_Ферма}

\end{document}
