\documentclass[12pt,a4paper]{scrartcl}
\usepackage[utf8]{inputenc}
\usepackage[english,russian]{babel}
\usepackage{indentfirst}
\usepackage{misccorr}
\usepackage{graphicx}
\usepackage{amsmath}
\usepackage{hyperref}
\usepackage{float}

\begin{document}

\section{"Странные" множители}

Ряд модулей можно разложить, потому что простые числа, которые его образуют, могут иметь какие-нибудь 'интересные' свойства, или же просто модуль может быть достаточно слабым для того, чтобы его факторизовать умным методом 'грубой силы'. Рассмотрим алгоритмы, которые позволяют факторизовать модуль RSA. 

\end{document}
