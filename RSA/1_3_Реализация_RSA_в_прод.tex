\documentclass[12pt,a4paper]{scrartcl}
\usepackage[utf8]{inputenc}
\usepackage[english,russian]{babel}
\usepackage{indentfirst}
\usepackage{misccorr}
\usepackage{graphicx}
\usepackage{amsmath}
\usepackage{hyperref}
\usepackage{float}
\usepackage{minted} %для вставки кода в документ

\begin{document}

\section{Библиотеки для реализации алгоритма RSA в продакшне}

В данной статье мы старались приводить все реализации на языке sage, но если вам понадобится внедрять алгоритм RSA в ваше приложение, то лучше не писать костыль, а обратиться к готовым библиотекам: то есть либо использовать библиотеку pycryptodome для python3, либо можно воспользоваться утилитой openssl.

Приведём примеры скриптов на языках python3 и bash-script с использованием библиотеки pycryptodome или утилиты openssl соответственно:\\

\textbf{Python3-реализация c использованием \href{https://yatb.kksctf.ru/}{OAEP}:}\\

\inputminted[tabsize=4,obeytabs,fontsize=\footnotesize]{python3}{./RSA_scripts/example_pycrypto.py}

\textbf{Bash-реализация:}\\

\inputminted[tabsize=4,obeytabs,fontsize=\footnotesize]{python3}{./RSA_scripts/openssl.sh}

\textbf{Источник:}

\href{https://www.youtube.com/watch?v=ibfb7\_-\_CGg\&list=PLLguubeCGWoaGFEDzduGmBhEgZ62p-Jqv\&index=6}{https://www.youtube.com/watch?...\&index=6}

\href{http://secgroup.dais.unive.it/wp-content/uploads/2012/11/Practical-Padding-Oracle-Attacks-on-RSA.html}{http://secgroup.dais.unive.it/...Oracle-Attacks-on-RSA.html}

\end{document}
