\documentclass[12pt,a4paper]{scrartcl}
\usepackage[utf8]{inputenc}
\usepackage[english,russian]{babel}
\usepackage{indentfirst}
\usepackage{misccorr}
\usepackage{graphicx}
\usepackage{amsmath}
\usepackage{hyperref}
\usepackage{float}

\begin{document}

\section{Слабые ГПСЧ}

Иногда в задачках слабость криптографии может скрываться генераторах случайных чисел, которыми генерируются простые числа для модуля $N$. Более подробно данный класс уязвимостей будет рассмотрен в разделе, посвящённом ГПСЧ, но сейчас рассмотрим слабый ГСПЧ применительно к шифрованию RSA.

Приведём в пример задачку с одного из ctf-соревнований VKA-ctf 2020:

\href{https://github.com/Red-Cadets/VKACTF-2020/blob/master/categories/crypto/Cry-1e-rsa_report/solve/README.md}{https://github.com/Red-Cadets/VKACTF-2020/.../solve/README.md}\\

Данный пример показывает, как числа p и q могут быть случайно сгенерированы близкими к степени простого числа (в данном случае близких к степени двойки) - это можно понять по близости к степени двойки модуля $N$.

Выяснить это можно следующим образом: 

\begin{enumerate}
	\item мы можем заметить, что $log2(N) = 1040$ почти без отклонений
	\item можем предположить, что $N = (2^{a} + b) * (2^{c} + d)$
	\item Знаем, что a + c = 1040, тогда b =  31, d = 55 - наименьшие из возможных при таких степенях двойки
\end{enumerate}

В общем случае вместо двойки, как уже было упомянуто выше, может быть иная константа, но двойка по понятным техническим причинам (двоичная система счисления в компьютерах) встречается чаще остальных чисел.

\end{document}
