\documentclass[12pt,a4paper]{scrartcl}
\usepackage[utf8]{inputenc}
\usepackage[english,russian]{babel}
\usepackage{indentfirst}
\usepackage{misccorr}
\usepackage{graphicx}
\usepackage{amsmath}
\usepackage{hyperref}
\usepackage{float}
\usepackage{minted} %для вставки кода в документ

\begin{document}

\section{Атака Хастада (broadcast attack)}

Данная атака основана на допущении, при котором для разных модулей используется одинаковая открытая экспонента $e$ и одинаковое исходное сообщение, которое будет зашифровано: 

$$C_1 = M^e\ mod N_1$$
$$C_2 = M^e\ mod N_2$$
$$...$$
$$C_e = M^e\ mod N_e$$

Также все модули должны быть попарно взаимно простыми, то есть  НОД$(N_i, N_j) = 1,$ для всех $i\neq j$ - иначе бы мы смогли факторизовать какой-нибудь из модулей, просто проверив их НОД (всего модулей должно быть $e$ штук).

То есть если все абоненты зашифровали одно и то же сообщение (например, приветствие или синхронизационное сообщение) на своих параметрах, то мы можем это сообщение восстановить, даже не имея закрытую экспоненту.

После составления системы:

$$\begin{cases} C_1 = M^e\ mod N_1 \\ C_2 = M^e\ mod N_2 \\ ... \\ C_e = M^e\ mod N_e \end{cases}$$

при помощи \href{https://yatb.kksctf.ru/}{китайской теоремы об остатках} мы находим решение системы сравнений:

$$C' = M^e\ mod\ N_1N_2...N_e$$

Так как система RSA предполагает, что $M < N_i$ для любого $i$, то $M^e << N_1N_2...N_e$.  В силу этих обстоятельств мы можем извлечь обычный корень $e$-ой степени из $M^e$, получив при этом исходное сообщение.\\

\textbf{Sage-реализация:}

\inputminted[tabsize=4,obeytabs,fontsize=\footnotesize]{python3}{./RSA_scripts/hastad.sage}

\textbf{Источник:}

\href{https://crypto.stanford.edu/~dabo/papers/RSA-survey.pdf}{https://crypto.stanford.edu/~dabo/papers/RSA-survey.pdf}

\href{https://ctf-wiki.github.io/ctf-wiki/crypto/asymmetric/rsa/rsa_coppersmith_attack-zh/}{https://ctf-wiki.github.io/ctf-wiki/crypto/asymmetric/rsa/rsa\_coppersmith\_attack-zh/}

\end{document}
