\documentclass[12pt,a4paper]{scrartcl}
\usepackage[utf8]{inputenc}
\usepackage[english,russian]{babel}
\usepackage{indentfirst}
\usepackage{misccorr}
\usepackage{graphicx}
\usepackage{amsmath}
\usepackage{hyperref}
\usepackage{float}
\usepackage{minted} %для вставки кода в документ

\begin{document}

\section{Атака Копперсмита}

Атака Копперсмита заключается в том, что мы можем дешифровать сообщение, если мы знаем его часть: например, мы знаем, что зашифрованное сообщение имеет примерный вид "the password is : <тут пишется пароль>". Например, \textit{"the password is : cupcake"}.

То есть мы можем представить наше сообщение как $m=m_0+x_0$, где $m_0$ известно, а $x_0$ неизвестен.

Атака представляет из себя нахождение решения $x$ полинома следующего вида:

$$f(x)=(m_0+x)^e-c,\text{ где } f(x_0)=0\ (mod\ N) $$

Атака Копперсмита сможет найти нам решение $f(x) \equiv 0\ (mod\ N)$, где $|x_0| \leq N^{\dfrac{1}{e}}$.

То есть мы сможем восстановить $\dfrac{1}{e}$ неизвестную часть сообщения, если остальное известно. Данное уравнение можно решить \href{https://yatb.kksctf.ru/}{методом Копперсмита}.\\

\textbf{Sage-реализация:}

\inputminted[tabsize=4,obeytabs,fontsize=\footnotesize]{python3}{./RSA_scripts/koppersmith.sage}

\textbf{Источники:}

Alexander May \textit{Using LLL-Reduction for Solving RSA and Factorization Problems}

\href{https://github.com/mimoo/RSA-and-LLL-attacks/blob/master/survey_final.pdf}{https://github.com/mimoo/RSA-and-LLL-attacks/blob/master/survey\_final.pdf}

\href{https://www.di.ens.fr/~fouque/ens-rennes/coppersmith.pdf}{D.Coppersmith.Small solutions to polynomial equations, and low exponent RSA vulnerabilities. Journal of Cryptology, 10:233-260, 1997.}

\end{document}
