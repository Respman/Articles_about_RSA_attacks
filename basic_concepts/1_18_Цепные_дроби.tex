\documentclass[12pt,a4paper]{scrartcl}
\usepackage[utf8]{inputenc}
\usepackage[english,russian]{babel}
\usepackage{indentfirst}
\usepackage{misccorr}
\usepackage{graphicx}
\usepackage{amsmath}
\usepackage{hyperref}
\usepackage{float}
\usepackage{tikz}

\begin{document}
	
\section{Цепные дроби}

Непрерывная дробь (или цепная дробь) — это конечное или бесконечное математическое выражение вида

$$[a_{0};a_{1},a_{2},a_{3},\cdots ]=a_{0}+{\cfrac {1}{a_{1}+{\cfrac {1}{a_{2}+{\cfrac {1}{a_{3}+\ldots }}}}}}\;$$

где $a_{0}$ есть целое число, а все остальные $a_n$ чисел — натуральные числа (положительные целые). При этом числа $a_{0},a_{1},a_{2},a_{3},...$ называются 'неполными частными' или элементами цепной дроби.\\

\textbf{Разложение в цепную дробь}

Любое вещественное число $x$ может быть представлено (конечной или бесконечной, периодической или непериодической) цепной дробью $[a_{0};a_{1},a_{2},a_{3},...]$, где

$$a_{0}=\lfloor x\rfloor ,x_{0}=x-a_{0},\\
a_{1}=\left\lfloor {\frac {1}{x_{0}}}\right\rfloor ,x_{1}={\frac {1}{x_{0}}}-a_{1},\\
\dots \\
a_{n}=\left\lfloor {\frac {1}{x_{n-1}}}\right\rfloor ,x_{n}={\frac {1}{x_{n-1}}}-a_{n},\\
\dots,$$

при этом $\lfloor x\rfloor$ обозначает целую часть числа $x$.

Для рационального числа $x$ это разложение оборвётся по достижении нулевого $x_{n}$ для некоторого $n$. В этом случае $x$ представляется конечной цепной дробью $x=[a_{0};a_{1},... ,a_{n}]$. Эффективным алгоритмом для преобразования обычной дроби в цепную является  \href{https://yatb.kksctf.ru/}{алгоритм Евклида}. Представление рационального числа в виде непрерывной дроби неоднозначно: если приведённый здесь алгоритм даёт непрерывную дробь $[... ,a_{n}]$, то непрерывная дробь $[... ,a_{n}-1,1]$ соответствует тому же самому числу.

Для иррационального $x$ все величины $x_{n}$ будут ненулевыми и процесс разложения можно продолжать бесконечно. В этом случае $x$ представляется бесконечной цепной дробью $x=[a_{0};a_{1},a_{2},a_{3},... ]$. Если последовательность $[a_0; a_1, a_2, a_3,...]$ состоит из бесконечно повторяющегося набора одних и тех же чисел (периода), то цепная дробь называется периодической. Число представляется бесконечной периодической цепной дробью тогда и только тогда, когда оно является квадратичной иррациональностью, то есть иррациональным корнем квадратного уравнения с целыми коэффициентами.\\

\textbf{Источник:}

\href{https://ru.wikipedia.org/wiki/%D0%9D%D0%B5%D0%BF%D1%80%D0%B5%D1%80%D1%8B%D0%B2%D0%BD%D0%B0%D1%8F_%D0%B4%D1%80%D0%BE%D0%B1%D1%8C}{https://ru.wikipedia.org/wiki/Непрерывная\_дробь}
	
\end{document}
