\documentclass[12pt,a4paper]{scrartcl}
\usepackage[utf8]{inputenc}
\usepackage[english,russian]{babel}
\usepackage{indentfirst}
\usepackage{misccorr}
\usepackage{graphicx}
\usepackage{amsmath}
\usepackage{hyperref}
\usepackage{float}
\usepackage{tikz}

\begin{document}
	
\section{Алгоритм двоичного умножения}

Рассмотрим пример умножения двоичных чисел в ЭВМ.

Выполнить умножение чисел $A = 10111_2$ и $B = 101_2$ в двоичной системе счисления.

Умножение выполняется в несколько этапов:

\begin{enumerate}
	
	\item Впишем множимое $A$, допустим, в 8-ми разрядный регистр, начиная с младших разрядов (нумерация разрядов начинается с нуля). В недостающие разряды записываем нули.
	
	\begin{table}[H]
		\centering
		\begin{tabular}{ccccccccc}
			Разр. & 7 & 6 & 5 & 4 & 3 & 2 & 1 & 0 \\
			A:    & 0 & 0 & 0 & 1 & 0 & 1 & 1 & 1
		\end{tabular}
	\end{table}
	
	\item Впишем множитель $B$ в 8-ми разрядный регистр, начиная с младших разрядов. В недостающие разряды записываем нули.
	
	\begin{table}[H]
		\centering
		\begin{tabular}{ccccccccc}
			Разр. & 7 & 6 & 5 & 4 & 3 & 2 & 1 & 0 \\
			B:    & 0 & 0 & 0 & 0 & 0 & 1 & 0 & 1
		\end{tabular}
	\end{table}
	
	\item Подготовим (обнулим) регистр результата $C$ удвоенной разрядности (16 бит). Произведение содержит в два раза больше разрядов чем исходные сомножители.
	
	\begin{table}[H]
		\centering
		\begin{tabular}{ccccccccccccccccc}
			Разр. & 15 & 14 & 13 & 12 & 11 & 10 & 9 & 8 & 7 & 6 & 5 & 4 & 3 & 2 & 1 & 0 \\
			C:    & 0  & 0  & 0  & 0  & 0  & 0  & 0 & 0 & 0 & 0 & 0 & 0 & 0 & 0 & 0 & 0
		\end{tabular}
	\end{table}
	
	\item Дальше выполняется следующий цикл:
	
	\subitem Анализируем очередной разряд множителя $B$ (начинаем с младших), если он '1', то прибавляем множимое $A$ к старшим разрядам регистра $C$, результат снова в $C$. Если очередной разряд множителя '0', пропускаем данный шаг.
	\subitem Сдвигаем содержимое регистра $C$ на один разряд вправо. При этом крайний левый (старший) разряд заполняется нулём. Но если перед этим была операция сложения, во время которой возник перенос из старшего разряда, то тогда крайний левый разряд заполняется единицей.
	
	\subitem Действия, описанные в п. п. 4.1 и 4.2, повторяются до тех пор, пока не будут проанализированы все разряды множителя.
	
	В итоге процесс умножения выглядит следующим образом:
	
	\begin{table}[H]
		\centering
		\begin{tabular}{ccccccccccccccccccc}
			& Разр. & с & 15 & 14 & 13 & 12 & 11 & 10 & 9 & 8 & 7 & 6 & 5 & 4 & 3 & 2 & 1 & 0 \\
			B: &       &   & 0  & 0  & 0  & 0  & 0  & 0  & 0 & 0 & 0 & 0 & 0 & 0 & 0 & 0 & 0 & 0 \\
			1  & +A    &   & 0  & 0  & 0  & 1  & 0  & 1  & 1 & 1 &   &   &   &   &   &   &   &   \\
			&       &   & -  & -  & -  & -  & -  & -  & - & - & - & - & - & - & - & - & - & - \\
			& =     &   & 0  & 0  & 0  & 1  & 0  & 1  & 1 & 1 & 0 & 0 & 0 & 0 & 0 & 0 & 0 & 0 \\
			& -->   &   & 0  & 0  & 0  & 0  & 1  & 0  & 1 & 1 & 1 & 0 & 0 & 0 & 0 & 0 & 0 & 0 \\
			&       &   &    &    &    &    & 1  & 1  & 1 &   &   &   &   &   &   &   &   &   \\
			0  & -->   &   & 0  & 0  & 0  & 0  & 0  & 1  & 0 & 1 & 1 & 1 & 0 & 0 & 0 & 0 & 0 & 0 \\
			1  & 1+A   &   & 0  & 0  & 0  & 1  & 0  & 1  & 1 & 1 &   &   &   &   &   &   &   &   \\
			&       &   & -  & -  & -  & -  & -  & -  & - & - & - & - & - & - & - & - & - & - \\
			& =     &   & 0  & 0  & 0  & 1  & 1  & 1  & 0 & 0 & 1 & 1 & 0 & 0 & 0 & 0 & 0 & 0 \\
			& -->   &   & 0  & 0  & 0  & 0  & 1  & 1  & 1 & 0 & 0 & 1 & 1 & 0 & 0 & 0 & 0 & 0 \\
			0  & -->   &   & 0  & 0  & 0  & 0  & 0  & 1  & 1 & 1 & 0 & 0 & 1 & 1 & 0 & 0 & 0 & 0 \\
			0  & -->   &   & 0  & 0  & 0  & 0  & 0  & 0  & 1 & 1 & 1 & 0 & 0 & 1 & 1 & 0 & 0 & 0 \\
			0  & -->   &   & 0  & 0  & 0  & 0  & 0  & 0  & 0 & 1 & 1 & 1 & 0 & 0 & 1 & 1 & 0 & 0 \\
			0  & -->   &   & 0  & 0  & 0  & 0  & 0  & 0  & 0 & 0 & 1 & 1 & 1 & 0 & 0 & 1 & 1 & 0 \\
			0  & -->   &   & 0  & 0  & 0  & 0  & 0  & 0  & 0 & 0 & 0 & 1 & 1 & 1 & 0 & 0 & 1 & 1
		\end{tabular}
	\end{table}
	
	где «+А» - число $A$ прибавляется к регистру $C$; «-->» - содержимое регистра $C$ сдвигается на один разряд вправо; «=» - показывается значение частичного произведения полученного после сложения, которое заносится в регистр $C$.
	
	\item Определяется знак результата. Если знаки исходных сомножителей одинаковы, то результирующее произведение положительно и наоборот. В данном случае знаки совпадают, следовательно результирующее произведение положительно.
	
\end{enumerate}
В итоге получается ответ:

$$10111_2\times 101_2 = 1110011_2$$

\textbf{Источник:}

\href{https://pandia.ru/text/80/277/92501.php}{https://pandia.ru/text/80/277/92501.php}
	
\end{document}
