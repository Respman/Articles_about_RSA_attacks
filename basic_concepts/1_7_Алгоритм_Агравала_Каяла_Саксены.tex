\documentclass[12pt,a4paper]{scrartcl}
\usepackage[utf8]{inputenc}
\usepackage[english,russian]{babel}
\usepackage{indentfirst}
\usepackage{misccorr}
\usepackage{graphicx}
\usepackage{amsmath}
\usepackage{hyperref}
\usepackage{float}
\usepackage{tikz}

\begin{document}
	
\section{Алгоритм Агравала - Каяла - Саксены}

Рассмотрим алгоритм Агравала - Каяла - Саксены детерминированной проверки числа на простоту:

\textbf{Ввод: } $n \in \mathbb{N},\ n>1$
\begin{enumerate}
	\item Если $n=a^{b}$ $a \in \mathbb{N}$ и целого $b > 1$, вернуть «составное».
	\item Найдем наименьшее $r$, такое что $o_{r}(n)>(log_{2}(n))^{2}$.
	\item Если $\text{НОД} ( a , n ) \neq 1$ для любого $a \leq r,$ вернуть "составное".
	\item Если для всех $a$ от $1$ до $\lfloor \sqrt{r} log\ n \rfloor$ верно, что $(x+a)^{n} \equiv x^{n}+a\ (mod\ \phi(x), n)$, вернуть "простое".
	\item Иначе вернуть "составное".
\end{enumerate}

Здесь  $o_{r}(n)$ обозначает показатель числа $n$  по модулю $r$ (то есть $n^{o_{r}(n)} \equiv 1\ (mod\ r)$), $\phi(x)$ - функция Эйлера.

Вычислительная сложность этого алгоритма — $O(log^{6} n)$. \\

\textbf{Источник:}

\href{https://ru.wikipedia.org/wiki/%D0%A2%D0%B5%D1%81%D1%82_%D0%90%D0%B3%D1%80%D0%B0%D0%B2%D0%B0%D0%BB%D0%B0_%E2%80%94_%D0%9A%D0%B0%D1%8F%D0%BB%D0%B0_%E2%80%94_%D0%A1%D0%B0%D0%BA%D1%81%D0%B5%D0%BD%D1%8B}{https://ru.wikipedia.org/wiki/Тест\_Агравала\_—\_Каяла\_—\_Саксены}\\

	
\end{document}
