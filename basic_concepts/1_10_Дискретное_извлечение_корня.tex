\documentclass[12pt,a4paper]{scrartcl}
\usepackage[utf8]{inputenc}
\usepackage[english,russian]{babel}
\usepackage{indentfirst}
\usepackage{misccorr}
\usepackage{graphicx}
\usepackage{amsmath}
\usepackage{hyperref}
\usepackage{float}
\usepackage{tikz}

\begin{document}
	
\section{Дискретное извлечение корня}

Задача дискретного извлечения корня звучит следующим образом: по данным $n$ ($n$ — простое), $a$ и $k$ требуется найти все $x$, удовлетворяющие условию:

$$x^k \equiv a \pmod{n}$$

\textbf{Алгоритм решения}

Решать задачу будем сведением её к задаче дискретного логарифмирования.

Для этого применим понятие первообразного корня по модулю $n$. Пусть $g$ — первообразный корень по модулю $n$ (первообразный корень - это такой элемент, который при возведении его в степень $k < \phi(n)$ переходит во все другие элементы поля, кроме себя; так как в нашей задаче число $n$ — простое, то его можно \href{https://yatb.kksctf.ru/}{найти}).

Отбросим сразу случай, когда $a=0$ — в этом случае сразу находим ответ $x=0$.

Поскольку в данном случае ($n$ — простое) любое число от $1$ до $n-1$ представимо в виде степени первообразного корня, то задачу дискретного корня мы можем представить в виде:

$${\left( g^y \right)}^k \equiv a \pmod{n}$$

где

$$x \equiv g^y \pmod{n}$$

Тривиальным преобразованием получаем:

$${\left( g^k \right)}^y \equiv a \pmod{n}$$

Здесь искомой величиной является $y$ - таким образом мы пришли к задаче дискретного логарифмирования в чистом виде. Эту задачу можно решить \href{https://yatb.kksctf.ru/}{алгоритмом baby-step-giant-step} Шэнкса, т.е. найти одно из решений $y_0$ этого уравнения (или обнаружить, что это уравнение решений не имеет).

Пусть мы нашли некоторое решение $y_0$ этого уравнения, тогда одним из решений задачи дискретного корня будет $x_0 = g^{y_0} \pmod{n}$.\\

\textbf{Источник:}

\href{https://e-maxx.ru/algo/discrete_root}{https://e-maxx.ru/algo/discrete\_root}
	
\end{document}
