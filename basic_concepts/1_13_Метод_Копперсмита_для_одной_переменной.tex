\documentclass[12pt,a4paper]{scrartcl}
\usepackage[utf8]{inputenc}
\usepackage[english,russian]{babel}
\usepackage{indentfirst}
\usepackage{misccorr}
\usepackage{graphicx}
\usepackage{amsmath}
\usepackage{hyperref}
\usepackage{float}
\usepackage{tikz}

\begin{document}
	
\section{Алгоритм Копперсмита для многочленов над конечным полем от одной переменной}

Изначально Копперсмит (Coppersmith) предложил алгоритм для решения уравнений с 'маленькими' корнями над конечным модулем. Через несколько лет Ховгрэв-Грэкхам (Howgrave-Graham) упростил его алгоритм.

Копперсмит предоставил алгоритм, который решает следующую задачу:


Пусть $N$ целое число с неизвестной факторизацией, которое имеет делитель $b \geq N^{\beta}$, $0 < \beta \leq 1$. Пусть $f(x)$ - многочлен от одной переменной со степенью, равной $\delta$ и пусть $c \geq 1$.\\
В таком случае мы сможем за время $\mathcal{O}(c\delta^5log^9(N))$ найти все решения $x_0$ выражения

\[ f(x) = 0 \pmod{b} \hspace{2mm} \text{ при } \hspace{2mm} |x_0| \leq c \cdot N^{\frac{\beta^2}{\delta}} \]


В нашем случае мы предположим, что $c=1$ и $\beta=1$, тогда мы сможем найти все решения для нашего предыдущего выражения, если $|x_0| \leq N^{\frac{1}{\delta}}$. Данный метод очень похож на атаку Хастада (когда мы предполагали, что если $M < N$, то и $M^e < N_1N_2...N_e$, поэтому мы могли вычислить обычный корень $e$-той степени из $M^e$).

Как только мы получим уравнение над полем целых чисел, то мы сможем воспользоваться одним из алгоритмов для решения уравнений над полем целых чисел (весь этот метод делался потому что для конечных колец при больших модулях существующие алгоритмы не эффективны). Ниже представлена теорема Копперсмита наглядно ($f(x_0)$ - многочлен, полученный уже после применения LLL-алгоритма):

\begin{center}
	\begin{tikzpicture}
		\node [above] at (2,1) {$f(x_0) = 0 \pmod{N}$, где $|x_0| < X$};
		\draw [<-,purple] (0,0) -- (0,1);
		\node [below] at (0.6,0) {$g(x_0) = 0$ над $\mathbb{Z}$};
	\end{tikzpicture}
\end{center}

Алгоритм, который привёл Копперсмит, очень сложен, потому что в нем использовался исходный многочлен, чтобы построить матрицу с относительно сложной структурой, после чего на её основе сокращались коэффициенты исходного многочлена, чтобы они удовлетворяли условиям теоремы Копперсмита.

Но Ховгрэйв-Грэкхэм предложил более элегантное решение: мы можем использовать не исходный многочлен для построения матрицы, а сгенерировать некоторое необходимое количество многочленов, которые будут иметь тот же корень, что и исходный (по построению), и уже на их основе строить матрицу, которую потом можно редуцировать. Самый короткий вектор в этой матрице (самый первый) будет иметь наименьшую длину, поэтому он будет иметь наименьшие коэффициенты и будет лучше всего подходить для того, чтобы у нас была возможность перевести его на поле целых чисел.

Мы будем пользоваться теоремой Ховгрэва-Грэкхэма (Howgrave-Graham):
Пусть $g(x)$ - многочлен от одной переменной с $n$ мономами, а $m$ - целое положительное число. Тогда теорема утверждает, что:

\begin{enumerate}
	\item $g(x_0) = 0 \pmod{N^m} \hspace{2mm} \text{ где } \hspace{2mm} |x_0| \leq X$
	\item $\|g(xX)\| < \frac{_{N^m}}{^{\sqrt{n}}}$
\end{enumerate}

В таком случае можно находить корни $g(x_0)=0$ над полем целых чисел.

Схематически это можно изобразить следующим образом:

\begin{center}
	\begin{tikzpicture}
		\node [above] at (2,4) {$f(x_0) = 0 \pmod{N}$, где $|x_0| < X$};
		\draw [<-] (0,0) -- (0,4);
		\node [below] at (0.6,0) {$g(x_0) = 0$ над $\mathbb{Z}$};
		\draw [purple,<-] (0,1) -- (1,1);
		\node [right] at (1.2,1.35) {$g(x_0) = 0 \pmod{N^m}$};
		\node [right] at (1.2,.65) {$\|g(xX)\| < \frac{N^m}{\sqrt{n}}$};
		\draw [purple,<-] (3,2) -- (3,4);
	\end{tikzpicture}
\end{center}

Теперь давайте конкретизируем построение многочлена $g(x_0) = 0\ (mod\ N^m)$.

Для этого нам нужно построить некоторое количество многочленов $f_i$ (мы будем называть их $g_{i,j}$ и $h_i$), которые будут иметь $x_0$ как корень(надо помнить, что $\delta$ является степенью многочлена $f$):

\begin{align*}
	g_{i,j}(x) &= x^j \cdot N^i \cdot f^{m-i}(x) \text{ при } i = 0,\hdots,m-1,\hspace{2mm} j=0,\hdots,\delta-1\\
	h_i(x) &= x^i \cdot f^m(x) \text{ при } i = 0,\hdots,t-1
\end{align*}

Эти полиномы имеют два свойства:
\begin{itemize}
	\item они имеют $x_0$ как корень, но уже по модулю $N^m$
	\item На каждой итерации конструируется новый моном - это позволяет добиться линейной независимости между векторами, а также позволяет создать матрицу нижнетреугольного вида (определитель такой матрицы легко посчитать в общем виде)
\end{itemize}

Если вы не понимаете, почему все многочлены $f_i$ имеют $x_0$ как корень, вспомните о том, что если $f(x_0) = 0 \pmod{N}$, то мы знаем что $f(x_0) = k \cdot N$\\

Теперь мы просто создаём матрицу, используя многочлены $f_i(xX)$ как строки матрицы.\\

При этом нам необходимо, чтобы соблюдалось условие

$$det(L) < N^{m \cdot n}$$

Иногда бывают случаи, когда мы не знаем значение модуля $N$, по которому нам нужно решить наше уравнение, но мы знаем его оценку снизу (например, если мы хотим решить уравнение по модулю $p$, который является делителем числа $N=pq$, при этом нам известно, что $p \geq N^\beta$). В таком случае мы должны проверять определитель полученной матрицы следующим образом:

$$det(L) < N^{m \cdot n \cdot \beta}$$


В нашем описании алгоритма всё ещё фигурируют неопределённые величины. 

Копперсмит в своей работе доказал, что если использовать верхнюю границу $X = \frac{1}{2} N^{\frac{\beta^2}{\delta}-\epsilon}$, то для определения наших параметров мы можем использовать следующие выражения:

\[ \begin{cases}
	0 < \epsilon \leq \frac{1}{7} \beta\\
	m = \left\lfloor \frac{\beta^2}{\delta \epsilon}\right\rfloor\\
	t = \left\lfloor \delta m (\frac{1}{\beta}-1)\right\rfloor\\
\end{cases} \]

Если вдруг получилось так, что определитель матрицы не удовлетворяет указанному неравенству, то мы должны увеличить $m$ на единичку и рассчитать матрицу заново (повторять это действие в случае необходимости, пока неравенство не будет выполнено).


После этого применяем для неё <LLL-алгоритм>, и получаем следующую матрицу:

$$B'=\begin{pmatrix}
	b_1=g(xX)\\
	b_2\\
	...\\
	b_n\\
\end{pmatrix}$$

Самый короткий вектор $B'$ (LLL-редуцированного базиса) $b_1$ должен стать коэффициентами многочлена $g(xX)$, из которого мы собираем многочлен $g(x)$. \\

После этого мы получаем многочлен над полем $N^m$:

$$g(x_0)=0\ (mod\ N^m),\ \parallel g(xX) \parallel < \dfrac{N^m}{\sqrt{n}}$$ 

Как следствие из теоремы Ховгрэйва-Грэкхэма, получаем $g(x_0)= 0$ над $\mathbb{Z}$

Далее находим корни уравнения с помощью \href{https://yatb.kksctf.ru/}{леммы Гензеля}: мы находим корни многочлена над конечным полем с помощью \href{https://yatb.kksctf.ru/}{алгоритма Берлекемпа}, а потом "поднимаем" их по лемме Гензеля и подставляем в исходное уравнение над целыми числами (можно использовать любой другой алгоритм численного решения уравнений).

В итоге получаем все корни $g(x)$ над $\mathbb{Z}$, которые будут и корнями исходного уравнения над конечным модулем.\\

Давайте ещё раз схематично изобразим алгоритм:\\

\begin{center}
	\begin{tikzpicture}
		% 1
		\node [above right] at (.5,10) {$f(x_0) = 0 \pmod{N}$, где $|x_0| < X$};
		\draw [<-] (1,0) -- (1,10);
		\node [below right] at (0.5,0) {$g(x_0) = 0$ над $\mathbb{Z}$};
		% 2
		\draw [<-] (1,1) -- (1.5,1);
		\node [right] at (1.7,1.35) {$g(x_0) = 0 \pmod{N^m}$};
		\node [right] at (1.7,.65) {$\|g(xX)\| < \frac{N^m}{\sqrt{n}}$};
		\draw [<-] (3.5,2) -- (3.5,10);
		% new 3
		\draw [purple,->] (6,10) -- (6,9.5);
		\node [below right] at (4,9.3) {собираем многочлены $f_i$ s.t. $f_i(x_0) = 0 \pmod{N^m}$};
		
		\draw [purple,->] (6,8.6) -- (6,7.8);
		\node [below right] at (4,7.5) {$B = \begin{pmatrix}
				f_i(xX) \\
				\vdots  \\
				f_n(xX)
			\end{pmatrix}$};
		
		\draw [purple,->] (6,5.5) -- (6,4.4);
		\node [right] at (6.3,5) {LLL};
		
		\node [below right] at (4,4.1) {$B' = \begin{pmatrix}
				b_1 = g(xX) \\
				b_2\\
				\vdots  \\
				b_n
			\end{pmatrix}$};
		
		\draw [<-,purple] (3.5,3) -- (4,3);
	\end{tikzpicture}\\
\end{center}

\textbf{Источники:}

\href{https://eprint.iacr.org/2013/512.pdf}{https://eprint.iacr.org/2013/512.pdf}

Alexander May \textit{Using LLL-Reduction for Solving RSA and Factorization Problems}

\href{https://en.wikipedia.org/wiki/Coppersmith%27s_attack#Generalizations}{https://en.wikipedia.org/wiki/Coppersmith\%27s\_attack\#Generalizations}
	
\href{https://trac.sagemath.org/ticket/1757}{https://trac.sagemath.org/ticket/1757}
	
\href{https://doc.sagemath.org/html/en/reference/polynomial_rings/sage/rings/polynomial/polynomial_modn_dense_ntl.html?highlight=small_roots#sage.rings.polynomial.polynomial_modn_dense_ntl.small_roots}{https://doc.sagemath.org/...polynomial\_modn\_dense\_ntl.small\_roots}
	
	
\href{https://github.com/mimoo/RSA-and-LLL-attacks}{https://github.com/mimoo/RSA-and-LLL-attacks}

\end{document}
