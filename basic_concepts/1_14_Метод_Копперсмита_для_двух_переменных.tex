\documentclass[12pt,a4paper]{scrartcl}
\usepackage[utf8]{inputenc}
\usepackage[english,russian]{babel}
\usepackage{indentfirst}
\usepackage{misccorr}
\usepackage{graphicx}
\usepackage{amsmath}
\usepackage{hyperref}
\usepackage{float}
\usepackage{tikz}

\begin{document}
	
\section{Решение уравнений от двух переменных с "маленькими" корнями методом Копперсмита}

Теперь рассмотрим несколько иную интерпретацию алгоритма Копперсмита, которая лучше (понятнее) ложится на случай с двумя переменными. Данный метод тоже опирается на теорему Ховгрэйва-Грэкхэма.

Пусть мы имеем многочлен $f(x,y),\ deg(f(x))=d$, при этом многочлен будет иметь хотя бы один моном старшей степени вида $x^ay^{d-a}$.

Пусть пара целочисленных значений $(x_0, y_0)$ будут являться конями нашего уравнения: $f(x_0,y,_0)\equiv 0\ (mod\ M)$

Пусть нам известны верхние оценки этих корней $|x_0|<X,\ |y_0|<Y$.

Теперь мы можем собрать матрицу из многочленов вида $g_{ijk}(xX,yY)=x^iy^j\left(\dfrac{f(x)}{M}\right)^k$, подставив в данное выражение тройки 

$$(i,j,k) \in S = \left\{ (i,j,k) \in \mathbb{Z}^3 |\ i+j+kd \leq md\text{ и }(i,j,k \geq 0)\text{ и }(i<a\text{ или }j<d-a)\right\}$$

После этого мы собираем все возможные мономы, которые есть в этих многочленах, и раскладываем по ним каждый многочлен как вектор.

Посчитаем определитель полученной матрицы:

$$det(L)=(XY)^{\dfrac{d^3}{6}m^3+o(m^3)}M^{-\dfrac{d^2}{6}m^3+o(m^3)} \approx (XY)^{\dfrac{d^3}{6}m^3}M^{-\dfrac{d^2}{6}m^3}$$

Чтобы полученные после применения LLL-алгоритма вектора удовлетворяли теореме Ховгрэйва-Грэкхема, нам нужно, чтобы выполнялось неравенство:

$$det(L)<2^{-\dfrac{\omega(\omega-1)}{4}}(\omega)^{-\dfrac{\omega-1}{2}}M^{-m}$$

, где $\omega=O(m^2d^2)$ - размер стороны квадратной матрицы. То есть мы получаем следующее выражение, из которого мы сможем найти  подходящее значение для параметра $m>0$:

$$(XY)^{\dfrac{d^3}{6}m^3}M^{-\dfrac{d^2}{6}m^2} < 2^{-\dfrac{m^2d^2(m^2d^2-1)}{4}}(m^2d^2)^{-\dfrac{m^2d^2-1}{2}}$$

После нахождения подходящего $m$ мы собираем матрицу и применяем к ней LLL-алгоритм. 

Мы получим матрицу новых векторов $h(xX,yY)$. Нам нужно собрать многочлены, исключив из множителей компоненты $X$ и $Y$ в тех степенях, в которых они представлены в мономах в изначальных многочленах.

Потом берём два вектора, которые при взаимной подстановке дают многочлен степени выше 1, и решаем получившееся после подстановки уравнение от одной переменной над полем целых чисел. Находим корни этого уравнения с помощью любого метода, предназначенного для нахождения корней над полем целых чисел - получаем либо корень $x_0$, либо корень $y_0$, после чего подставляем его в один из двух векторов и решаем получившееся уравнение уже от другой переменной над целыми числами, получаем недостающий корень.

\textbf{Источники:}

\href{http://theory.stanford.edu/~gdurf/durfee-thesis-phd.pdf}{http://theory.stanford.edu/~gdurf/durfee-thesis-phd.pdf} - начиная с 25 стр

\href{https://github.com/KonstT-math/Durfee-Boneh_RSA/blob/master/boneh_durfee.sage}{https://github.com/KonstT-math/Durfee-Boneh\_RSA/blob/master/boneh\_durfee.sage}
	
\end{document}
