\documentclass[12pt,a4paper]{scrartcl}
\usepackage[utf8]{inputenc}
\usepackage[english,russian]{babel}
\usepackage{indentfirst}
\usepackage{misccorr}
\usepackage{graphicx}
\usepackage{amsmath}
\usepackage{hyperref}
\usepackage{float}
\usepackage{tikz}

\begin{document}
	
\section{Расширенный алгоритм Евклида}

В то время как 'обычный' алгоритм Евклида просто находит наибольший общий делитель двух чисел $a$ и $b$, расширенный алгоритм Евклида находит помимо НОД находит также коэффициенты $x$ и $y$ такие, что:

$$a \cdot x + b \cdot y = \text{НОД } (a, b).$$

То есть он находит коэффициенты, с помощью которых НОД двух чисел выражается через сами эти числа.\\

\textbf{Алгоритм}

Внести вычисление этих коэффициентов в алгоритм Евклида несложно, достаточно вывести формулы, по которым они меняются при переходе от пары $(a,b)$ к паре $((b\ mod\ a),a)$.

Итак, пусть мы нашли решение $(x_1,y_1)$ задачи для новой пары $((b\ mod\ a),a)$:

$$(b\ mod\ a) \cdot x_1 + a \cdot y_1 = g,$$

и хотим получить решение $(x,y)$ для нашей пары $(a,b)$:

$$a \cdot x + b \cdot y = g.$$

Для этого преобразуем величину $b\ mod\ a$:

$$(b\ mod\ a) = b - \left\lfloor \frac{b}{a} \right\rfloor * a.$$

Подставим это в приведённое выше выражение с $x_1$ и $y_1$ и получим:

$$g = (b\ mod\ a) \cdot x_1 + a \cdot y_1 = \left( b - \left\lfloor \frac{b}{a} \right\rfloor *a \right) * x_1 + a*y_1,$$

и, выполняя перегруппировку слагаемых, получаем:

$$g = b \cdot x_1 + a \cdot \left( y_1 - \left\lfloor \frac{b}{a} \right\rfloor *x_1 \right)$$

Сравнивая это с исходным выражением над неизвестными x и y, получаем требуемые выражения:

$$\begin{cases}
	x = y_1 - \left\lfloor \frac{b}{a} \right\rfloor*x_1 \\
	y=x_1
\end{cases}$$

\textbf{Источник:}

\href{https://e-maxx.ru/algo/extended_euclid_algorithm}{https://e-maxx.ru/algo/extended\_euclid\_algorithm}
	
\end{document}
