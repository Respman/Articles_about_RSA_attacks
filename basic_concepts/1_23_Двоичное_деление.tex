\documentclass[12pt,a4paper]{scrartcl}
\usepackage[utf8]{inputenc}
\usepackage[english,russian]{babel}
\usepackage{indentfirst}
\usepackage{misccorr}
\usepackage{graphicx}
\usepackage{amsmath}
\usepackage{hyperref}
\usepackage{float}
\usepackage{tikz}

\begin{document}
	
\section{Алгоритм двоичного деления}

Данный алгоритм можно использовать и для приведения числа по модулю. 

Рассмотрим процесс деления двоичных чисел в ЭВМ на примере деления числа $A=10010111_2$ на $B=101_2$, который включает следующие этапы:

\begin{enumerate}
	
	\item Впишем делимое $A$ в 16-ти разрядный регистр, начиная с младших разрядов (нумерация разрядов начинается с нуля). В недостающие разряды записываем нули.
	
	\begin{table}[H]
		\centering
		\begin{tabular}{ccccccccccccccccc}
			Разр. & 15 & 14 & 13 & 12 & 11 & 10 & 9 & 8 & 7 & 6 & 5 & 4 & 3 & 2 & 1 & 0 \\
			A:    & 0  & 0  & 0  & 0  & 0  & 0  & 0 & 0 & 1 & 0 & 0 & 1 & 0 & 1 & 1 & 1
		\end{tabular}
	\end{table}
	
	Так как для выполнения деления требуется производить операцию вычитания, это требует использования знаковой арифметики. И поэтому в данном случае 15-й разряд является знаковым (0 - соответствует положительному числу, 1 - отрицательному), а старшим разрядом числа является 14-й разряд.
	
	\item Впишем делитель $B$ в 16-ти разрядный регистр, начиная с младших разрядов. В недостающие разряды записываем нули.
	
	\begin{table}[H]
		\centering
		\begin{tabular}{ccccccccccccccccc}
			Разр. & 15 & 14 & 13 & 12 & 11 & 10 & 9 & 8 & 7 & 6 & 5 & 4 & 3 & 2 & 1 & 0 \\
			B:    & 0  & 0  & 0  & 0  & 0  & 0  & 0 & 0 & 0 & 0 & 0 & 0 & 0 & 1 & 0 & 1
		\end{tabular}
	\end{table}
	
	Здесь также, как и с числом $A$ 15-й разряд является знаковым, а старшим разрядом числа является 14-й разряд. Эти знаковые разряды будут показывать нам знаки, образующихся в процессе деления, а также знаки частичных остатков. Они не имеют никакого отношения к знакам исходных операндов и знаку результата, а играют чисто технологическую роль.
	
	\item Предварительный сдвиг делителя. Сдвинем делитель $B$ влево так, чтобы позиция старшей значащей единицы, в нем, совпала с позицией старшей значащей единицы в делимом $A$. Количество необходимых для этого сдвигов запомним в числе $k$. В нашем случае старшая значащая единица в делимом $A$ расположена в 7-м разряде, a в делителе $B$ - в 2-м разряде. Следовательно, необходимо сдвинуть число $B$ влево на 5 разрядов ($k = 5$).
	Сдвинутый делитель выглядит следующим образом :
	
	\begin{table}[H]
		\centering
		\begin{tabular}{ccccccccccccccccc}
			Разр. & 15 & 14 & 13 & 12 & 11 & 10 & 9 & 8 & 7 & 6 & 5 & 4 & 3 & 2 & 1 & 0 \\
			B:    & 0  & 0  & 0  & 0  & 0  & 0  & 0 & 0 & 1 & 0 & 1 & 0 & 0 & 0 & 0 & 0
		\end{tabular}
	\end{table}
	
	\item Так как в процессе деления множитель $B$ придётся не только прибавлять но и вычитать, то нам необходимо иметь число $-B$. Для этого представим $B$ в дополнительном коде.
	
	\begin{table}[H]
		\centering
		\begin{tabular}{ccccccccccccccccc}
			B:    & 1  & 1  & 1  & 1  & 1  & 1  & 1 & 1 & 0 & 1 & 1 & 0 & 0 & 0 & 0 & 0
		\end{tabular}
	\end{table}
	
	\item Процесс деления будет следующий:
	
	\subitem Вычитаем из делимого $A$ делитель $B$ (т. е. прибавляем $-B$).
	
	\subitem Анализируем знак полученного частичного остатка (15-й разряд). В регистр результата записываем '0', если остаток отрицательный и единицу в противном случае. Помним, что отрицательному числу соответствует наличие единицы в 15-м разряде и наоборот.
	
	\subitem Сдвигаем частичный остаток на один разряд влево. При этом крайний правый (младший) разряд заполняется нулем, а знаковый разряд (15-й) в процессе сдвига не участвует.
	
	\subitem Прибавляем к частичному остатку делитель $B$ если остаток отрицательный, либо вычитаем делитель в противном случае.
	
	\subitem Анализируем знак полученного частичного остатка (15-й разряд). В регистр результата записываем '0' если остаток отрицательный и единицу противном случае.
	
	\subitem Действия, описанные в пунктах 5.3-5.5, выполняем $k$ раз (если $k=0$, то ни разу не выполняем). Но, если после очередной операции сложения/вычитания частичный остаток, по модулю, будет меньше чем исходный (несдвинутый) делитель, то операция деления прекращается, а частное дополняется нулями так, чтобы число разрядов частного равнялось $k+1$.
	
	В итоге процесс деления для вышеуказанного примера будет выглядеть следующим образом:
	
	\begin{table}[H]
		\centering
		\begin{tabular}{ccccccccccccccccccc}
			& Разр. & с & 15 & 14 & 13 & 12 & 11 & 10 & 9 & 8 & 7 & 6 & 5 & 4 & 3 & 2 & 1 & 0 \\
			Частное & A     &   & 0  & 0  & 0  & 0  & 0  & 0  & 0 & 0 & 1 & 0 & 0 & 1 & 0 & 1 & 1 & 1 \\
			& -B    &   & 1  & 1  & 1  & 1  & 1  & 1  & 1 & 1 & 0 & 1 & 1 & 0 & 0 & 0 & 0 & 0 \\
			&       &   & -  & -  & -  & -  & -  & -  & - & - & - & - & - & - & - & - & - & - \\
			0       & =     &   & 1  & 1  & 1  & 1  & 1  & 1  & 1 & 1 & 1 & 1 & 1 & 1 & 0 & 1 & 1 & 1 \\
			&       & 1 & 1  & 1  & 1  & 1  & 1  & 1  & 1 & 1 & 1 & 1 &   &   &   &   &   &   \\
			& <--   &   & 1  & 1  & 1  & 1  & 1  & 1  & 1 & 1 & 1 & 1 & 1 & 0 & 1 & 1 & 1 & 0 \\
			& +B    &   & 0  & 0  & 0  & 0  & 0  & 0  & 0 & 0 & 1 & 0 & 1 & 0 & 0 & 0 & 0 & 0 \\
			&       &   & -  & -  & -  & -  & -  & -  & - & - & - & - & - & - & - & - & - & - \\
			1       & =     &   & 0  & 0  & 0  & 0  & 0  & 0  & 0 & 0 & 1 & 0 & 0 & 0 & 1 & 1 & 1 & 0 \\
			&       & 1 & 1  & 1  & 1  & 1  & 1  & 1  & 1 &   &   &   &   &   &   &   &   &   \\
			& <--   &   & 0  & 0  & 0  & 0  & 0  & 0  & 0 & 1 & 0 & 0 & 0 & 1 & 1 & 1 & 0 & 0 \\
			& -B    &   & 1  & 1  & 1  & 1  & 1  & 1  & 1 & 1 & 0 & 1 & 1 & 0 & 0 & 0 & 0 & 0 \\
			&       &   & -  & -  & -  & -  & -  & -  & - & - & - & - & - & - & - & - & - & - \\
			1       & =     &   & 0  & 0  & 0  & 0  & 0  & 0  & 0 & 0 & 0 & 1 & 1 & 1 & 1 & 1 & 0 & 0 \\
			&       & 1 & 1  & 1  & 1  & 1  & 1  & 1  & 1 & 1 & 1 & 1 &   &   &   &   &   &   \\
			& <--   &   & 0  & 0  & 0  & 0  & 0  & 0  & 0 & 0 & 1 & 1 & 1 & 1 & 1 & 0 & 0 & 0 \\
			& -B    &   & 1  & 1  & 1  & 1  & 1  & 1  & 1 & 1 & 0 & 1 & 1 & 0 & 0 & 0 & 0 & 0 \\
			&       &   & -  & -  & -  & -  & -  & -  & - & - & - & - & - & - & - & - & - & - \\
			1       & =     &   & 0  & 0  & 0  & 0  & 0  & 0  & 0 & 0 & 0 & 0 & 1 & 0 & 1 & 1 & 0 & 0 \\
			&       & 1 & 1  & 1  & 1  & 1  & 1  & 1  & 1 & 1 & 1 & 1 & 1 &   &   &   &   &   \\
			& <--   &   & 0  & 0  & 0  & 0  & 0  & 0  & 0 & 0 & 0 & 1 & 0 & 1 & 1 & 0 & 0 & 0 \\
			& -B    &   & 1  & 1  & 1  & 1  & 1  & 1  & 1 & 1 & 0 & 1 & 1 & 0 & 0 & 0 & 0 & 0 \\
			&       &   & -  & -  & -  & -  & -  & -  & - & - & - & - & - & - & - & - & - & - \\
			1       & =     &   & 0  & 0  & 0  & 0  & 0  & 0  & 0 & 0 & 0 & 0 & 0 & 1 & 0 & 0 & 0 & 0 \\
			&       &   &    &    &    &    &    &    &   &   & 1 & 1 &   &   &   &   &   &   \\
			& <--   &   & 0  & 0  & 0  & 0  & 0  & 0  & 0 & 0 & 0 & 0 & 1 & 0 & 0 & 0 & 0 & 0 \\
			& -B    &   & 1  & 1  & 1  & 1  & 1  & 1  & 1 & 1 & 0 & 1 & 1 & 0 & 0 & 0 & 0 & 0 \\
			&       &   & -  & -  & -  & -  & -  & -  & - & - & - & - & - & - & - & - & - & - \\
			0       & =     &   & 1  & 1  & 1  & 1  & 1  & 1  & 1 & 1 & 1 & 0 & 0 & 0 & 0 & 0 & 0 & 0
		\end{tabular}
	\end{table}
	
	где «+B» - делитель $B$ прибавляет к регистру делимого $A$; «-B» - делитель $B$ вычитается из регистра делимого $A$; «<--» - частичный остаток сдвигается на один разряд влево; «=» - показывается значение частичного остатка полученного после сложения.
	
	\item Определяем остаток от деления. Для этого анализируем последний частичный остаток. В нашем случае он равен $1111111110000000$.
	
	\subitem Анализируем знак остатка (15-й разряд). В знаковом разряде содержится единица, значит требуется коррекция остатка. Для коррекции прибавим к нему делитель $B$.
	
	\begin{table}[H]
		\centering
		\begin{tabular}{cccccccccccccccccc}
			Разр. & с  & 15 & 14 & 13 & 12 & 11 & 10 & 9 & 8 & 7 & 6 & 5 & 4 & 3 & 2 & 1 & 0                     \\
			& 1   & 1  & 1  & 1  & 1  & 1  & 1  & 1 & 1 &   &   &   &   &   &   &   &   \\
			&     & 1  & 1  & 1  & 1  & 1  & 1  & 1 & 1 & 1 & 0 & 0 & 0 & 0 & 0 & 0 & 0 \\
			+B    &    & 0  & 0  & 0  & 0  & 0  & 0  & 0 & 0 & 1 & 0 & 1 & 0 & 0 & 0 & 0 & 0                     \\
			&     & -  & -  & -  & -  & -  & -  & - & - & - & - & - & - & - & - & - & -                     \\
			=     &    & 0  & 0  & 0  & 0  & 0  & 0  & 0 & 0 & 0 & 0 & 0 & 0 & 0 & 0 & 0 & 0                    
		\end{tabular}
	\end{table}
	
	\subitem Так как в процессе деления частичные остатки были сдвинуты 5 раз влево, то для получения верного значения последний полученный остаток необходимо сдвинуть 5 раз вправо (вернуть на место). После сдвига имеем:
	
	\begin{table}[H]
		\centering
		\begin{tabular}{ccccccccccccccccc}
			Разр. & 15 & 14 & 13 & 12 & 11 & 10 & 9 & 8 & 7 & 6 & 5 & 4 & 3 & 2 & 1 & 0 \\
			& 0  & 0  & 0  & 0  & 0  & 0  & 0 & 0 & 0 & 0 & 0 & 0 & 0 & 0 & 0 & 1
		\end{tabular}
	\end{table}
	
	\item Определяем знак результата. Если знаки исходных операндов одинаковы, то результирующее частное положительно и наоборот. В данном случае знаки совпадают, следовательно результирующее частное положительно.
	
\end{enumerate}

В итоге получается ответ:

$$10010111_2  / 101_2 = 111110_2$$

и $1_2$ в остатке.\\

\textbf{Источник:}

\href{https://pandia.ru/text/80/277/92501.php}{https://pandia.ru/text/80/277/92501.php}
	
\end{document}
