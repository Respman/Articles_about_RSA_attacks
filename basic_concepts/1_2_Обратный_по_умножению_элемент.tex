\documentclass[12pt,a4paper]{scrartcl}
\usepackage[utf8]{inputenc}
\usepackage[english,russian]{babel}
\usepackage{indentfirst}
\usepackage{misccorr}
\usepackage{graphicx}
\usepackage{amsmath}
\usepackage{hyperref}
\usepackage{float}
\usepackage{tikz}

\begin{document}
	
\section{Нахождение обратного по умножению элемента в конечном поле}

При нахождении обратного по умножению элемента для элемента $e$ в конечном поле по модулю $N$ сперва запускается \href{https://yatb.kksctf.ru/}{расширенный алгоритм Евклида} для $(N, e)$. Он выдаёт тройку $(1, i, j)$, где $in+je\equiv 1 (mod\ N)$. Отсюда следует $je \equiv 1 (mod\ N)$, и поэтому $j=e^{-1}=d$. 
	
\end{document}
