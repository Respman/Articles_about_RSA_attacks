\documentclass[12pt,a4paper]{scrartcl}
\usepackage[utf8]{inputenc}
\usepackage[english,russian]{babel}
\usepackage{indentfirst}
\usepackage{misccorr}
\usepackage{graphicx}
\usepackage{amsmath}
\usepackage{hyperref}
\usepackage{float}
\usepackage{tikz}

\begin{document}
	
\section{Дисперсия случайной величины}

Дисперсия случайной величины — мера разброса значений случайной величины относительно её математического ожидания. Обозначается $D[X]$ в русской литературе и $\operatorname {Var}(X)$ (англ. variance) в зарубежной. В статистике часто употребляется обозначение $\sigma _{X}^{2}$ или $\sigma ^{2}$. 

Дисперсией случайной величины называют \href{https://yatb.kksctf.ru/}{математическое ожидание} квадрата отклонения случайной величины от её математического ожидания.

Пусть $X$ — случайная величина, определённая на некотором вероятностном пространстве. Тогда дисперсией называется

$$D[X]=M\left[{\big (}X-M[X]{\big )}^{2}\right],$$

где символ $M$ обозначает математическое ожидание. 

\textbf{Источник:}

\href{https://ru.wikipedia.org/wiki/%D0%94%D0%B8%D1%81%D0%BF%D0%B5%D1%80%D1%81%D0%B8%D1%8F_%D1%81%D0%BB%D1%83%D1%87%D0%B0%D0%B9%D0%BD%D0%BE%D0%B9_%D0%B2%D0%B5%D0%BB%D0%B8%D1%87%D0%B8%D0%BD%D1%8B}{https://ru.wikipedia.org/wiki/Дисперсия\_случайной\_величины}
	
\end{document}
