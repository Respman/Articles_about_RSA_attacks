\documentclass[12pt,a4paper]{scrartcl}
\usepackage[utf8]{inputenc}
\usepackage[english,russian]{babel}
\usepackage{indentfirst}
\usepackage{misccorr}
\usepackage{graphicx}
\usepackage{amsmath}
\usepackage{hyperref}
\usepackage{float}
\usepackage{tikz}

\begin{document}
	
\section{Алгоритм Евклида}

Алгоритм Евклида для целых чисел:

Пусть $a$ и $b$ — целые числа, не равные одновременно нулю, и последовательность чисел

$$a>b>r_{1}>r_{2}>r_{3}>r_{4}>\ \dots \ >r_{n}$$

определена тем, что каждое $r_{k}$ — это остаток от деления предпредыдущего числа на предыдущее, а предпоследнее делится на последнее нацело, то есть:

$$a=bq_{0}+r_{1},$$
$$b=r_{1}q_{1}+r_{2},$$
$$r_{1}=r_{2}q_{2}+r_{3},$$
$$\cdots $$
$$r_{k-2}=r_{k-1}q_{k-1}+r_{k},$$
$$\cdots $$
$$r_{n-2}=r_{n-1}q_{n-1}+r_{n},$$
$$r_{n-1}=r_{n}q_{n}.$$

Тогда НОД$(a, b)$, наибольший общий делитель $a$ и $b$, равен $r_n$, последнему ненулевому члену этой последовательности. 

\textbf{Пример}

Для примера найдём НОД $a = 1071$ и $b = 462$. Для начала из $1071$ вычтем максимальное кратное значение $462$. Мы должны дважды отнять $462$\ $(q_0 = 2)$ - в результате получим остаток $147$:

$$1071 = 2 \times 462 + 147.$$

Затем от $462$ отнимаем кратное значение $147$, пока не получим разность меньше, чем $147$. Мы должны трижды отнять $147\ (q_1 = 3)$, оставаясь с остатком $21$:

$$462 = 3 \times 147 + 21.$$

Затем от $147$ отнимаем кратное значение $21$, пока не получим разность меньше, чем $21$. Мы должны семь раз отнять $21\ (q_2 = 7)$, оставаясь в результате без остатка:

$$147 = 7 \times 21 + 0.$$

Таким образом последовательность $a > b > r_1 > r_2 > r_3 > ... > r_n$ в данном конкретном случае будет выглядеть так:

$$1071 > 462 > 147 > 21.$$


Так как последний остаток равен нулю, алгоритм заканчивается числом $21$ и НОД$(1071, 462) = 21$.

В табличной форме шаги будут выглядеть следующим образом:

\begin{table}[H]
	\centering
	\begin{tabular}{|c|c|c|}
		\hline
		\textbf{Шаг $k$} & \textbf{Равенство}     & \textbf{Частное и остаток}                             \\ \hline
		0                & $1071 = q_0\times 462 + r_0$ & \textit{$q_0 = 2 $ и $ r_0 = 147 $}                    \\ \hline
		1                & $462 = q_1\times 147 + r_1$  & \textit{$q_1 = 3$ и $r_1 = 21$}                        \\ \hline
		2                & $147 = q_2\times21 + r_2$   & \textit{$q_2 = 7$ и $r_2 = 0$; алгоритм заканчивается} \\ \hline
	\end{tabular}
\end{table}

Если требуется найти НОД для более чем двух чисел $(x_1,x_2,x_3,...)$, то сначала находится НОД $(x_1,x_2)$ для первых двух чисел, потом находится НОД для НОД $(x_1,x_2)$ и следующего числа $x_3$ и т.д., пока не получим одно число.\\ 

\textbf{Источник:}

\href{https://ru.wikipedia.org/wiki/%D0%90%D0%BB%D0%B3%D0%BE%D1%80%D0%B8%D1%82%D0%BC_%D0%95%D0%B2%D0%BA%D0%BB%D0%B8%D0%B4%D0%B0}{https://ru.wikipedia.org/wiki/Алгоритм\_Евклида}

	
\end{document}
