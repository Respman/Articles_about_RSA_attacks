\documentclass[12pt,a4paper]{scrartcl}
\usepackage[utf8]{inputenc}
\usepackage[english,russian]{babel}
\usepackage{indentfirst}
\usepackage{misccorr}
\usepackage{graphicx}
\usepackage{amsmath}
\usepackage{hyperref}
\usepackage{float}
\usepackage{tikz}

\begin{document}
	
\section{Критерий Эйлера}

Критерий Эйлера гласит, что если $n^{(p-1)/2} \equiv 1\ (mod\ p)$, то символ Лежандра $\left({n \over p}\right)=1$ и данное число $n$ является квадратичным вычетом по модулю $p$.

Если $n^{(p-1)/2} \equiv -1\ (mod\ p)$, то $\left({n \over p}\right)=-1$ и данное число $n$ является квадратичным невычетом по модулю $p$.\\

\textbf{Источник:}

\href{https://ru.wikipedia.org/wiki/%D0%9A%D1%80%D0%B8%D1%82%D0%B5%D1%80%D0%B8%D0%B9_%D0%AD%D0%B9%D0%BB%D0%B5%D1%80%D0%B0}{https://ru.wikipedia.org/wiki/Критерий\_Эйлера}
	
\end{document}
