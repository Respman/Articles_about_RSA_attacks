\documentclass[12pt,a4paper]{scrartcl}
\usepackage[utf8]{inputenc}
\usepackage[english,russian]{babel}
\usepackage{indentfirst}
\usepackage{misccorr}
\usepackage{graphicx}
\usepackage{amsmath}
\usepackage{hyperref}
\usepackage{float}
\usepackage{tikz}

\begin{document}
	
\section{Функция Эйлера}

Функция Эйлера $\phi(n)$ показывает количество чисел, не превышающих $n$ и взаимнопростых с $n$. Взаимнопростыми числами называют два числа, которые не имеют общих делителей, кроме единицы. Количество чисел, взаимнопростых с $p^k$, равно $\phi(p^k)=p^k-p^{k-1}$ (то есть все числа до $p^k$, кроме чисел $p,\ p^2,...,p^{k-1}$, которых $p^{k-1}$ штук. В частном случае, когда $k=1$, то $\phi(p)=p-1$ интерпретируется очень просто, потому что простое число взаимнопросто со всеми меньшими числами, кроме себя самого, по определению). Функция эйлера обладает мультипликативным свойством: $\phi(m*n)=\phi(m)*\phi(n)$ - если мы возьмём число $m'$, взаимнопростое с $m$ и $n'$, взаимнопростое с $n$, то и произведение $m'n'$ будет взаимнопросто с $mn$.

	
\end{document}
