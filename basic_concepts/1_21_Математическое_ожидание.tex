\documentclass[12pt,a4paper]{scrartcl}
\usepackage[utf8]{inputenc}
\usepackage[english,russian]{babel}
\usepackage{indentfirst}
\usepackage{misccorr}
\usepackage{graphicx}
\usepackage{amsmath}
\usepackage{hyperref}
\usepackage{float}
\usepackage{tikz}

\begin{document}
	
\section{Математическое ожидание случайной величины}

Математическое ожидание —  среднее (взвешенное по вероятностям возможных значений) значение случайной величины.

Обозначается ${\mathbb {E}}[X]$ (например, от англ. Expected value или нем. Erwartungswert); в русскоязычной литературе также встречается обозначение $M[X]$ (возможно, от англ. Mean value или нем. Mittelwert, а возможно от «Математическое ожидание»). В статистике часто используют обозначение $\mu$. 


Если $X$ — дискретная случайная величина, имеющая распределение

$$\mathbb {P} (X=x_{i})=p_{i},\;\sum \limits _{i=1}^{\infty }p_{i}=1,$$

то прямо из определения следует, что

$$\mathbb {E} [X]=\sum \limits _{i=1}^{\infty }x_{i}\,p_{i}.$$

\textbf{Источник:}

\href{https://ru.wikipedia.org/wiki/%D0%9C%D0%B0%D1%82%D0%B5%D0%BC%D0%B0%D1%82%D0%B8%D1%87%D0%B5%D1%81%D0%BA%D0%BE%D0%B5_%D0%BE%D0%B6%D0%B8%D0%B4%D0%B0%D0%BD%D0%B8%D0%B5}{https://ru.wikipedia.org/wiki/Математическое\_ожидание}
	
\end{document}
