\documentclass[12pt,a4paper]{scrartcl}
\usepackage[utf8]{inputenc}
\usepackage[english,russian]{babel}
\usepackage{indentfirst}
\usepackage{misccorr}
\usepackage{graphicx}
\usepackage{amsmath}
\usepackage{hyperref}
\usepackage{float}
\usepackage{tikz}

\begin{document}
	
\section{Алгоритм линейного подъема Гензеля}

Алгоритм линейного подъёма Гензеля (Hensel Linear-Lifting Algorithm) основан на лемме Гензеля и работает следующим образом - мы вычислили корни многочлена над небольшим конечным полем (например, с помощью \href{https://yatb.kksctf.ru/}{алгоритма Берлекемпа}), потом поднимаем корни многочлена до корней над полем целых чисел.\\

\textbf{Алгоритм}

\textbf{Вход:} Простое число $p$, натуральное число $k$, а также полиномы $p(x) \in \mathbb{Z}[x],\ g_1(x)$ и $h_1(x)$, оба $\in GF(p)[x]$, такие, что

$$p(x) \equiv g_1(x)h_1(x)\ (\mod p), \text{и НОД} (g_1(x),h_1(X))=1.$$

\textbf{Выход:} Полиномы $g_k(x)$ и $h_k(x)$, оба $\in (\mathbb{Z}/p^k)[x]$, такие, что $p(x) \equiv g_k(x)h_k(x)\ (\mod p^k),$ а также $g_k(x) \equiv g_1(x)\ (\mod p),$ и $h_k(x) \equiv h_1(x)\ (\mod p)$.

\begin{enumerate}
	\item \textit{Инициализация}. Положить $g(x) = g_1(x)$, $h(x) = h_1(x)$, а затем применить расширенный алгоритм Евклида в $GF(p)[x]$ к полиномам $g(x)$ и $h(x)$ и вычислить полиномы $a(x)$ и $b(x)$, оба $\in GF(p)[x]$, такие, что $g(x)a(x)+h(x)b(x)=1$ над $GF(p)$.
	
	\item \textit{Основной цикл}. Для $i=2,3,..,k$ выполнять \{вычислить поправку $c(x) := [p(x)-g(x)h(x)]/p^{i-1}\ (\mod p)$; затем применить \href{https://yatb.kksctf.ru/}{алгоритм решения полиномиального уравнения в $\mathbb{Z}_{p^j}[x]$} к $g(x),\ h(x), a(x), b(x)$, и $c(x)$, все $\in GF(p)[x]$, чтобы вычислить полиномы $a_{i-1}^{'}(x)$ и $b_{i-1}^{'}(x)$ такие, что $g(x)a_{i-1}^{'}(x)+h(x)b_{i-1}^{'}(x)=c(x)$ над $GP(p)$ и положить $g(x) := g(x) + p^{i-1}b_{i-1}^{'}(x)$  и $h(x) :=h(x) + p^{i-1}a_{i-1}^{'}(x)$ \}.
	
	\item \textit{Выход}. Вернуть $g(x)$ и $h(x)$.
\end{enumerate}

\textbf{Пример}

На вход подаётся линейная гензелева конструкция ($p(x)$ не обязательно нормирован). Рассмотрим полином $p(x)=112x^4+58x^3-31x^2+107x-66$ в $\mathbb{Z}[x]$ для которого имеет место сравнение

$$p(x) \equiv (8x^2+12x+10)(x^2+9x+9)\ (\mod 13).$$

Мы поднимем это разложение с $GF(13)$ до $\mathbb{Z}_{13^2},$ пользуясь линейной гензелевой конструкцией и арифметикой неотрицательных вычетов. Для лучшего понимания процесса сообщим, что над целыми числами имеет место разложение $p(x)=(8x^2+7x-6)(14x^2-5x+11)$; эти два сомножителя полинома $p(x)$ найдены путём применения алгоритма подъёма. Очевидно, что в нашем примере $g_1(x)=(8x^2+12x+10)$ и $h_1(x)=(x^2+9x+9)$. Пользуясь расширенным алгоритмом Евклида, вычисляем полиномы $a(x)=x+11$ и $b(x)=5x+11$ в $GF(13)[x]$, такие, что $g_1(x)a(x)+h_1(x)b(x)=1$ над $GF(13)$. [В действительности из расширенного алгоритма Евклида мы получаем $a(x)=5x+3$, $b(x)=5x+11$ но тогда $g_1(x)a(x)+h_1(x)b(x)=5$ в поле $GF(13)$, следовательно, мы должны подправить их, домножив на соответствующую константу.]

Затем из соотношения $p(x)-g_1(x)h_1(x)=13c(x)$ (где $g_1(x),\ h_1(x)$ рассматриваются в $\mathbb{Z}[x]$) получаем $c(x)=8x^4+11x^3+9x^2+6x+1$ над $GF(13)$. Заметим, что $p\cdot c(x)$ - это разность между $p(x)$ и $g_1(x)h_1(x)$ в $\mathbb{Z}_{p^2}[x]$, нам нужно теперь подкорректировать наши сомножители, т.е. прибавить к $g_1(x)$ и $h_1(x)$, чего им не хватает для того, чтобы разность исчезла. Используя  \href{https://yatb.kksctf.ru/}{алгоритм решения полиномиального уравнения в $\mathbb{Z}_{p^j}[x]$}, вычисляем полиномы $a'(x)=x+9$ и $b'=8x^2+12x+6$ в $GF(13)[x]$, такие, что $g_1(x)a'(x)+h_1(x)b'(x)=c(x)$ над $GF(13)$. Затем определяем

$$g_2(x)=g_1(x)+p\cdot b'(x)=(8x^2+12x+10)+13(8x^2+12x+6)=112x^2+168x+88,$$
$$h_2(x)=h_1(x)+p\cdot a'(x)=(x^2+9x+9)+13(x+9)=x^2+22x+126.$$

Здесь $p(x)=g_2(x)h_2(x)$ над $\mathbb{Z}_{169=13^2}$, легко может быть проверено, но ни ни $g_2(x)$, ни $h_2(x)$ не делят $p(x)$ в $\mathbb{Z}[x]$. Однако заметим, что $g_2(x)=8(14x^2+21x+11)$, и если мы скомбинируем сомножитель $8$ с $h_2(x)$ то получим $8\cdot h_2(x)=8(x^2+22x+126)=8x^2+176x+1008 \equiv 8x^2+7+163 \equiv 8x^2+7x-6\ (\mod 169)$. Здесь $(8x^2+7x-6)|p(x)$ над целыми числами, и, следовательно, это один из сомножителей; разделив, мы получаем второй сомножитель $14x^2-5x+11$.\\

\textbf{Источники:}

Акритас А. Основы компьютерной алгебры с приложениями. Интернет-ресурс:

\href{https://scask.ru/r_book_oca.php?id=79}{https://scask.ru/r\_book\_oca.php?id=79}, стр 426
	
\end{document}
